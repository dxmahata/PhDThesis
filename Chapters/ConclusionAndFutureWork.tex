% Chapter 9

\chapter{Conclusion and Future Work} % Main chapter title

\label{Conclusion} % For referencing the chapter elsewhere, use \ref{Chapter1} 

\lhead{Chapter 9. \emph{Conclusion and Future Work}} % This is for the header on each page - perhaps a shortened title
\doublespacing
\setlength{\parindent}{1cm}
\section{Conclusion}
This dissertation introduced the idea of `Event Identity Information Management' (EIIM) from textual content in social media and discussed how it could be used for tracking unstructured references related to real-life events having impressions in social media. It also introduced the Event Identity Information Management life cycle and explained its various components. It pointed to the novel contributions in each of the component and showed the effectiveness and performance of the devised techniques against the state-of-the-art baselines. 

The characteristics of informative and non-informative content produced in 3.8 million (approx) tweets during three real-life events in Twitter were studied. Cues were obtained from the analysis for identifying informative content. It was also observed that the supervised models used for assigning informativeness scores to tweets are generic and are not always well suited for identifying event-specific informative tweets. Moreover, they don't have the ability to simultaneously identify event-specific informative hashtags, text units, URLs and users. This created an intriguing scenario and led to the need of a model that identifies and ranks event-specific informative content from Twitter.

Using the cues from the analysis it was found that hashtags used for annotating tweets, text units used for expressing the tweet content, URLs shared for providing additional information and the users posting them during an event are the main units of information that could be leveraged for measuring event-specific informativeness.  Mutually reinforcing relationships were identified between the tweets, hashtags, text units, URLs and users posting them during an event, and their associations were represented in a graph structure that forms the underlying framework for the proposed ranking algorithm. This graph is named as \textit{EventIdentityInfoGraph}. The semantics of the relationships between the vertices of the graph were defined and quantified. Initial event-specific scores were assigned to the vertices. An algorithm - \textit{EventIdentityInfoRank}, was proposed for ranking the vertices. The algorithm makes use of the mutually reinforcing chains formed between the vertices of \textit{EventIdentityInfoGraph} for propagating the event-specific scores of a vertex to its neighboring vertices. The accumulated score of the vertices after the convergence of the algorithm is used for simultaneously ranking streams of tweets, hashtags, text units, URLs and users producing them during two real-life events in terms of their event-specific informativeness. Promising results were obtained using the proposed EIIM framework. The results were evaluated by comparing the performance of our approach with six other approaches including the state-of-the-art \textit{SeenRank} algorithm used by Seen for ranking tweets displayed in their website. The approach proposed in this dissertation outperformed all the baselines by large margins for NDCG@n and Precision@n scores proving it to be the most effective and robust algorithm for identifying event-specific informative content from noisy stream of tweets in Twitter.

The problem of discovering event-specific informative content in Twitter was solved by proposing a robust and scalable `Event Identity Information Management' framework that goes through a cycle of data processing pipeline, known as the EIIM Life Cycle.  Since the features used for the proposed techniques are commonly found in most of the social media platforms, it is assumed that the EIIM Life Cycle has the potential to produce effective results in other social media channels as well. The dissertation also pointed the ability of the framework to scale in a distributed processing environment. Some of the works that can be considered as future works of this dissertation are discussed next. 


\section{Future Work}

\subsection{Summarizing Event Related Content}
Given the huge amount of content produced in social media related to real-life events, summarization of the content can be very useful in such a scenario. It can help the users to overcome the problem of information overload. One of the most important characteristics of the summarization techniques is to identify the most salient units of information from the textual posts. The \textit{EventIdentityInfoGraph} and the \textit{EventIdentityInfoRank} algorithm as proposed and implemented in this dissertation can be tuned for such a purpose. One of the steps that needs to be taken is to find how the salient event identity information units obtained as an output of \textit{EventIdentityInfoRank} can be used for constructing event summaries from short textual social media messages. We have started looking into this problem and look forward to solve it using the framework as proposed in this dissertation. One of the main advantages and novelty in solving this problem using the EIIM framework would be the capability of generating event summaries as the event evolves. 

\subsection{Identifying Insightful Opinionated Content Related to Events}
Users often share insightful opinionated content about different topics, people, organizations, in social media. Such content is also generated in the context of an event. For example, in a sporting event the fans may post a lot of opinionated content about the players. Not all of them will be insightful. Similarly, in a product launch event, the prospective customers, or the reviewers may post very insightful and opinionated reviews about the new product. This type of content is extremely useful for the prospective customers, targeted marketing and for automated systems in order to identify the positive and negative aspects of the product that is creating buzz in social media. Identification of such insightful opinionated tweets can lead to the discovery of very useful and strategic information. On considering a mix of named entities and unigram opinionated words as text units in the \textit{EventIdentityInfoGraph} we obtained some preliminary encouraging results. A glimpse of the results obtained for a basketball game ''Miami Heats VS Cleveland Cavaliers", played on 25th December, 2014 is as follows:

Top 10 insightful and opinionated tweets for an hour related to the game
\begin{enumerate}

\item	Good win for the Heat tonight against Cavs and Lebron. Great game for Wade and Deng. Just imagine if Bosh were healthy. \#HeatvsCavs

\item	Good work Dwayne Wade. Good work Miami Heat. LeBron is embarrassed. It's all over his face. \#NBA \#heatvscavs

\item	Great game on Christmas Heat Showed up and spoiled Lebron Return to MIA! \#Wade County \#HeatvsCavs \#NBAChristmas

\item	Lebron leaves Miami high and dry and they cheer his return. Some even cheering cavs. Embarrassing bandwagon fan base. \#heatv…

\item	I totally understand LBJ move to Cleveland and like it. But if I'm a \#Miami fan, I would boo LeBron like crazy today. \#heatvscavs \#CLEvsMIA

\item	Stay classy \#Miami. Good game vs. Lebron and; Cavs. \#NBA \#MIAvsCLE \#HeatvsCavs \#Heat \#HeatNation

\item Loul Deng playing both ends of the floor. He's playing good D to LBJ \#heatvscavs

\item	Heat fans ; Cavs fans. Class vs no class. No burning a jersey in Miami \#heatvscavs \#HeatNation

\item	WE FUCKING WON!!!!!! LETS GO HEAT \#HEATgame \#HeatNation \#HeatvsCavs Wade with 31 points 5 assist 5 rebounds! Good shit MIAMI

\item	Kevin Love is overrated. Big fish, small pond in MN and injury prone. \#HeatvsCavs \#NBAXmas

\end{enumerate}

The above tweets point to the reactions of the viewers for the game as well as the players participating in the event. We plan to work on this and take steps to tune the EIIM framework in order to make it better than the state-of-the-art techniques, for identifying insightful opinionated content from social media. This useful content once identified and ranked can also be used for generating opinion summaries. 







\subsection{Event-specific Recommendations}
The graph based data structure used for storing the EIIS can be used for generating event related informative recommendations in near real-time. The graph structure aids in exploring relationships between tweets, text units, hashtags, users and URLs. Moreover, the `Event Identity Information Process' component processes the EIIS and assigns event-specific informativeness scores to its vertices. These scores combined with the relationships between the vertices can be leveraged for recommending users to other users who are producing event-specific informative content. Similarly, event-specific informative tweets, URLs and hashtags can be recommended. A naive approach has been implemented. For example following is a refined tweet recommendation for an event obtained from a snapshot of the \textit{EventIdentityInfoGraph} created for the event: “BlackLivesMatter”: Protest movement against the killing of Eric Garner.

\textbf{Original Tweet:}

\begin{itemize}
\item \#BREAKING \#NEWS | New York City Mayor Says, \#BlackLivesMatter \\ http://t.co/qYvp8L8gDh | \#BLACK  \@HCP520
\end{itemize}

\textbf{Recommended Tweets:}

\begin{itemize}

\item New York: What's the plan? Where are the protests happening tonight? \#EricGarner \#BlackLivesMatter \#MichaelBrown \#ICantBreathe

\item Brooklyn District Attorney to Convene Grand Jury in Case of \#AkaiGurley NBC New York http://t.co/mLlYPy39Pa \#BlackLivesMatter

\item New York Today! \#ShutItDown \#economicshutdown \#BlackLivesMatter \#ICantBreathe \#EricGarner \#nojusticenoprofits http://t.co/F0TrZtx2Y5

\end{itemize}

Similarly an user can get other recommended users who are talking on the same topic. Hashtags and topics can also be recommended. It can further lead to clustering of similar content and discovery of communities around different topics related to the event.


\subsection{Distributed Processing of EventIdentityInfoGraph}
The \textit{EventIdentityInfoRank} algorithm processes the nodes and edges of the \textit{EventIdentityInfoGraph} iteratively to come up with a simultaneous ranking of its heterogeneous vertices. The processing of the heterogeneous nodes can be distributed and then an aggregate score can be assigned to each vertex after an individual iteration. This is perfectly suitable for implementing the algorithm in a mapreduce paradigm. Similar steps are taken by the PageRank algorithm for ranking billions of web pages at scale. We plan to use the Giraph\footnote{http://giraph.apache.org/} distributed graph processing library on top of HDFS or use Apache Spark for implementing the process of ranking the vertices of \textit{EventIdentityInfoGraph}.

\subsection{Event Ontology for Social Media}
Another research direction than can be explored in the future is to develop  an ontology for representing the extracted event identity information units. This will enable a systematic categorization of the event identity information units into different concepts that can aid in formal reasoning. Reasoning on the relationships and characteristics of individual event identity information units can lead to extraction of deeper insights. For example, questions like ``who are the people involved?", ``what are the places mentioned in the event related content?", ``how are the different people and places related to one another?", and so on. Attempts have already been made on formulating ontological representation of the multimedia content produced during events by Troncy et al. \cite{troncy2010linking}, as well as, representing social media communities \cite{breslinsemantically}. Ontologies for integrating information from different types of documents have also been proposed \cite{doerr2006towards}, that can be used for representing the social media references and the relationships between the content extracted from them. Another ontology, which is of great interest to us is the Basic Formal Ontology (BFO) \cite{smith2002basic}. This is because of the fact that BFO is an upper level ontology and has constructs for all types of entities including events. It will give the freedom of exploring the way events can be defined in social media and the representation of its related textual content. Also, BFO considers events as separate from the other types of named entities like, person or a place. This enables reasoning about the relationships between several events with a person, and vice versa. At the same time relationships between the events can be explored.