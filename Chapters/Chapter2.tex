% Chapter 3

\chapter{Social Media and Real-life Events} % Main chapter title

\label{events} % For referencing the chapter elsewhere, use \ref{Chapter1} 

\lhead{Chapter 2. \emph{Social Media and Real-life Events}} % This is for the header on each page - perhaps a shortened title

\section{Social Media}
Social media is defined as a group of Internet-based applications that build on the ideological and technological foundations of Web 2.0, and that allow the creation and exchange of user-generated content \cite{kaplan2010users}. These Internet-based applications broadly ranges from blogs, microblogs, media sharing webistes, social bookmarking websites, social news to social networking websites. Brief description of the most popular types of social media websites is given below.

\subsection{Blogs}
A blog can be defined as a website that displays, in a reverse chronological order, the entries by one or more individuals and usually has links to comments on specific postings. Blogs often provide opinions, commentaries, or news on a particular subject, such as food, politics, or local news. Some of them also function as personal online diaries. Most of the time the entries of a blog is archived and is accessible at a later time. For the purpose of constant syndication, RSS or XML feeds for the blogs are made available. An individual entry in a blog is known as a blog post. A typical blog post can combine text, images and links to other blogs, web pages and other media related to its topic. The universe of all the blogs on the Internet is known as blogosphere \cite{agarwal2014time}.

\subsection{Microblogs}
Microblogs are similar to blogs, but a shorter version of it. Most of the microblogging websites pose limitations on the length of an individual post. Twitter, one of the most popular microblogging website has a limitation of 140 characters. This makes the textual posts in these platforms, extremely concise. Users often associate URLs that lead to external sources of information related to the posts. A post may also contain attached image or video. The microblogging services mostly focus on short updates that are pushed out to anyone subscribed to receive the updates. This is made possible by enabling the users to form directed networks of \textit{friends} and \textit{followers}. The \textit{followers} of an user are entitled to get all the updates posted by him. Mostly these updates are public. 

\subsection{Media Sharing}
Media sharing services allow its users to upload and share various multimedia content such as pictures and videos. Most services have additional social features such as profiles, commenting, etc. The most popular are Instagram, Pinterest, YouTube and Flickr. The media elements are often enriched with geographical and topical ``tags'' by the users who create them and the consumers who browse them. These tags acts as very useful meta data and allow automated programs to leverage them for efficient organization and retrieval of the videos and images that otherwise lack textual content. 

\subsection{Social Bookmarking}
These are the genre of social media services that allow its users to save, organize and manage links to various websites and resources around the internet. Most allow to ``tag" URLs for making them easy to search and share. The most popular are Delicious and StumbleUpon. Some of the services like StumbleUpon also allow their users to form friendship networks. These websites often provide different browsing experiences through interfaces that help the users to search for most recent tags, most popular ones, and so on.

\subsection{Social News}
Social news websites allow people to post various news items or links to articles that are external to the website, and then allows its users to cast their ``vote' on the items. The voting is the core social aspect as the items that get the most votes are displayed most prominently. This makes it an ideal crowdsourced news platform. It is up to the community of users to decide which news items gets seen by more people. Users can also ``tag'' the news stories and comment on them. The most popular are Digg and Reddit.

\subsection{Social Networking}
Social networking websites are the ones that allow its users to connect with each other and form networks. The connections are generally non-directional and reciprocal. Two users who are connected to each other are considered as \textit{friends}. Usually the users in these webistes have a profile that presents the personal information of the user as provided by him. The users have various ways to interact with other users, and also sometimes have the ability to set up groups. These social networks may be based on a certain theme such as interests, location, and profession. Facebook is the most popular personal social network and LinkedIn is the most popular professional network.

Some of the other types of websites that can also be categorized as social media services are, social messaging services, collaboration tools, rating or review sites, personal broadcasting tools, virtual worlds, and group buying. Table \ref{socialmediacat}, lists popular social media websites in different categories. Some of the websites may overlap and fall into multiple categories due to the broad range of services provided by them. For example, Facebook is not only a popular social networking website, but also a widely used social messaging service.

\begin{table}[h]
\centering
\caption{Popular social media websites belonging to different categories.}
\label{socialmediacat}
\begin{tabular}{|c|l|}
\hline
\textbf{Category} & \multicolumn{1}{c|}{\textbf{Popular Social Media Websites}} \\ \hline
\textbf{Blogs} & Blogger, Medium, Wordpress, Squarespace \\ \hline
\textbf{Microblogs} & Twitter, Tumblr, Posterous \\ \hline
\textbf{Media Sharing} & \begin{tabular}[c]{@{}l@{}}Flickr, Instagram, YouTube, Vimeo, Dailymotion, Metacafe,\\ Viddler, Pinterest\end{tabular} \\ \hline
\textbf{Social Bookmarking} & Delicious, StumbleUpon, Scoop, Slashdot \\ \hline
\textbf{Social News} & Digg, Reddit, Newsvine, Propeller \\ \hline
\textbf{Social Networking} & \begin{tabular}[c]{@{}l@{}}Facebook, Google Plus, LinkedIn, Ello, CafeMom,\\ Gather, Fitsugar\end{tabular} \\ \hline
\textbf{Virtual Worlds} & Second Life, World of Warcraft, Farmville \\ \hline
\textbf{Group Buying} & Groupon, Living Social, Crowdsavings \\ \hline
\textbf{Personal Broadcasting} & Blog Talk radio, Ustream, Livestream \\ \hline
\textbf{Review/Rating} & Amazon ratings, Angie’s List \\ \hline
\textbf{Collaboration Tools} & Wikipedia, WikiTravel, WikiBooks \\ \hline
\textbf{Social Messaging} & WhatsApp, Viber \\ \hline
\end{tabular}
\end{table}

According to Pew Research Center Facebook, LinkedIn, Pinterest, Instagram and Twitter are the top five most popular social media websites used by american adult Internet users\footnote{http://www.pewinternet.org/2015/01/09/social-media-update-2014/}. All the above social media websites exhibit certain common characteristics that is also responsible for their wide usage and huge popularity.

\begin{table}[h]
\centering
\caption{Number of active users for the top five social media websites used by american adults.}
\label{socialmediastat}
\begin{tabular}{|c|c|}
\hline
\textbf{Social Media Website} & \textbf{Number of Active Users} \\ \hline
\textbf{Facebook} & 1.31 billion \\ \hline
\textbf{LinkedIn} &  347 million \\ \hline
\textbf{Pinterest} & 70 million \\ \hline
\textbf{Instagram} & 100 million \\ \hline
\textbf{Twitter} & 289 million \\ \hline
\end{tabular}
\end{table}

\begin{enumerate}
\item \textbf{Accessibility:} Social media websites are freely available to whoever has an Internet connection. This makes these websites easily accessible all over the world. This is unlike the mainstream media or the print media, which people subscribe and buy in the form of magazines, newspapers, journals, etc. Also, the mainstream media can be easily controlled by the government that might lead to propagation of biased information. For example, during the ``Egyptian Revolution of 2011", the mainstream media was biased, regulated by the government, and did not portray the true picture of the situation in Egypt. On the other hand, it was social media through which people discussed about the actual atrocities of the government and grouped together for giving rise to the revolution \cite{hamdy2012framing}. 

\item \textbf{Permanence:}

\item \textbf{Reach:}


\item \textbf{Recency:}

\item \textbf{Usability:}
\end{enumerate}

\section{General Challenges in Social Media Mining}
\subsection{Information Overload}
A daily average of 58 million tweets is posted in Twitter\footnote{http://www.statisticbrain.com/twitter-statistics/}.On an average 60 million  photos are shared in Instagram daily\footnote{http://instagram.com/press/}. Facebook stores 300 petabytes  of data related to its users from all over the world\footnote{http://expandedramblings.com/index.php/by-the-numbers-17-amazing-facebook-stats/}. These are some compelling statistics that makes social media not only rich in volume of data, but also variety, and the velocity at which data is being generated. Due to the great pace at which data is produced in social media, the search engines and content filtering algorithms often face the problem of information overload \cite{hemp2009death}. They suffer from the dilemma of assessing the accuracy and quality of information content in the sources being produced over their freshness. Thus, collecting different types of references of entities from various social media platforms, assessing their quality, resolving and extracting identity information of the entities poses great challenges in such a situation.

\subsection{Veracity of Sources}
Judging the accuracy of the information and deciding relevant information content in social media references for the purpose of extracting entity identity attributes constitutes another challenging situation. For trending topics the search engines have started showing real-time feeds from social media websites in their search results. This has attracted spammers who post trending hash-tags or keywords along with their spam content in order to attract people to their websites offering products or services \cite{benevenuto2010detecting}. An alarming 355\% growth of social spam has been reported in 2013\footnote{http://www.likeable.com/blog/2013/11/10-surprising-social-media-statistics/}. Social media has also been instrumental in spreading misinformation and rumors. Spread of misinformation not only results in pandemonium among the users\footnote{http://www.theguardian.com/uk/interactive/2011/dec/07/london-riots-twitter}  but also result in extraction of completely wrong information about entities.

\subsection{Informal Text}
Unlike sources of news media and edited documents on the web, the textual content of the social media sources are highly colloquial and pose great difficulties in extracting information. One of the most important sources of information about events, prevalent in the domain of social media are the micro-blogging platforms. Micro blogs pose additional challenges due to their brevity, noisiness, idiosyncratic language, unusual structure and ambiguous representation of discourse \cite{bontcheva2013twitie}. Variation in language, less grammatical structure of sentences, unconventional uses of capitalization, frequent use of emoticons, and abbreviations have to be dealt by any system processing social media content. Moreover, various signals of communications embedded in the text in the form of hash-tags (eg.\#sochi), retweets (RT) and user mentions (@) should be understood by the system in order to extract the contextual information hidden in the text. Intentional misspellings sometimes demonstrate examples of intonation in written text \cite{prevost1996information}. For instance, expressions like, `this is so cooool', emphasizes stress on the emotions and conveys more information that should be captured. It has been shown that it is extremely challenging for the state-of-the art information extraction algorithms to perform efficiently and give accurate results for micro-blogs \cite{derczynski2013microblog}. For example, named entity recognition methods typically show 85-90\% accuracy on longer texts, but 30-50\% on tweets \cite{ritter2011named}. Status messages in social networking websites, content in question answering websites, reviews, and discussions in blogs, and forums exhibit similar nature and present similar challenges to information extraction and text mining procedures.



\subsection{Sampling Bias}
Most commonly used method for obtaining data samples from social media websites is by using their application programming interfaces (APIs). Given the humungous amounts of data produced in real-time, the APIs cannot provide all the data to every single API requests. The requests are often made through a query interface by passing certain query parameters to the APIs. The amount of data returned against the queries may vary. This depends upon the popularity of the content related to the query. For example, in Twitter studies have estimated that by using Twitter's Streaming API users can expect to receive anywhere from 1\% of the tweets to over 40\% of tweets in near real-time\footnote{https://www.brightplanet.com/2013/06/twitter-firehose-vs-twitter-api-whats-the-difference-and-why-should-you-care/}. The only way to get access to all the tweets is to buy the firehose service, which is seldom done for academic purposes. Other real-time social media publishing services mostly follow the same model. Therefore, this might lead to biasness in the samples collected for studying event related phenomenon and for tracking all the important event related information being produced in real-time.

\subsection{Multiple Data Sources}
The APIs (Application Programming Interfaces) of the different social media websites returns data in different formats (JSON, XML) using different web standards (REST, HTTPS). Moreover, the information obtained from a social media website is dependent upon the type of content it produces. A video sharing website might return an entirely different set of information from a blogging website. Thus, integrating the data obtained from the various social media platforms for the purpose of extraction and tracking of event related information is also one of the challenges.

\subsection{Lack of Evaluation Datasets}
There is a lack of ground truth evaluation data for most of the social media text mining tasks. In traditional data mining research, there is often two types of datasets. One of them is known as training dataset and the other is known as test dataset. The models are trained or developed using the training datasets and are evaluated on test datasets. Thus, the test datasets act as the ground truth. The test dataset for various text mining tasks is mostly not available for social media data. It is often the duty of the researchers to create new test datasets in order to solve a specific task in social media. Sometimes this data might not be a benchmark dataset due to various unwanted noise and human error or perception in annotating the data. This might lead to wrong assumptions and false results.




\section{Events from Different Perspectives}

\subsection{Topic Detection and Tracking}

\subsection{Automatic Content Extraction}

\subsection{Multimedia Event Detection}

\section{Events in Social Media}

\section{Problem of EIIM in Social Media}