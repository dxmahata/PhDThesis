% Chapter 6

\chapter{Applications} % Main chapter title

\label{applications} % For referencing the chapter elsewhere, use \ref{Chapter1} 

\lhead{Chapter 6. \emph{Applications}} % This is for the header on each page - perhaps a shortened title

\section{Event Monitoring and Analysis}
References related to real-life events are extremely abundant in social media. Right from natural disasters such as the `Haiti Earthquake' to international sporting events like the `Winter Olympics' to socio-political and socio-economical events that shook the world such as presidential elections, `Egyptian Revolution', and recessions were covered, analyzed, extrapolated and informed by social media. This prolific event-specific content in social media makes it a promising ground for performing event analytics. Platforms like Geofeedia\footnote{http://geofeedia.com/}, TwitterStand\footnote{http://twitterstand.umiacs.umd.edu/}, Twitris\footnote{http://twitris.knoesis.org/}, Truthy\footnote{http://truthy.indiana.edu/}, and TweetTracker\footnote{http://tweettracker.fulton.asu.edu/}  have developed techniques to provide analytics related to different local and global real-life events. 

Monitoring social media has become one of the essential activities of national security agencies for predicting potential threats and mass protests \cite{ghannam2011social}. Social media is being used for tracking terrorism activities \cite{oh2011information}, collective actions \cite{agarwal2014online}, and countering cyber-attack threats\footnote{https://www.recordedfuture.com/}. One of the main components of each of these applications is tracking references related to the events. The proposed EIIM model could be an essential component of such systems. It would help in identifying, tracking and analyzing events and its related references in an organized manner over time.





\section{Event Information Retrieval}

\section{Event Opinion Mining}

\section{Event-specific Recommendations}

\section{Event Management and Marketing}

\section{Social Media Data Integration}
Organizations have increasingly started integrating the data available in social media with the enterprise data\footnote{http://www.altimetergroup.com/research/reports/social-data-intelligence}. Social media data is most powerful when it is combined with daily transactional data and the master data to give a comprehensive view of customers, products and business conditions. Customers often openly talk about the products in social media and build communities around hashtags related to different products. An EIIM system capable of operating in social media could go a long way in collecting the right information about the entities of concern maintained in the enterprise databases and integrate the collected information with the already existing ones. The entity resolution aspect would further help in managing the data quality issues related to data integration. In such conditions the EIIM model proposed could be used for integrating entity information from two distinct domains of enterprise system and social media in order to gain strategic intelligence related to business of an organization. This would further help an organization in marketing, corporate communications, public relations, customer support, product development, advertising, market research, product recommendations and gaining competitive intelligence.
