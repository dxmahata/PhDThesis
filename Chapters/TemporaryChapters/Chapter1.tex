% Chapter 1

\chapter{Introduction} % Main chapter title

\label{introduction} % For referencing the chapter elsewhere, use \ref{Chapter1} 

\lhead{Chapter 1. \emph{Introduction}} % This is for the header on each page - perhaps a shortened title

\section{Social Media and Real-life Events}

\section{General Challenges in Mining Social Media Text}

\subsection{Information Overload}
A daily average of 58 million tweets is posted in Twitter\footnote{http://www.statisticbrain.com/twitter-statistics/}.On an average 60 million  photos are shared in Instagram daily\footnote{http://instagram.com/press/}. Facebook stores 300 petabytes  of data related to its users from all over the world\footnote{http://expandedramblings.com/index.php/by-the-numbers-17-amazing-facebook-stats/}. These are some compelling statistics that makes social media not only rich in volume of data, but also variety, and the velocity at which data is being generated. Due to the great pace at which data is produced in social media, the search engines and content filtering algorithms often face the problem of information overload \cite{hemp2009death}. They suffer from the dilemma of assessing the accuracy and quality of information content in the sources being produced over their freshness. Thus, collecting different types of references of entities from various social media platforms, assessing their quality, resolving and extracting identity information of the entities poses great challenges in such a situation.

\subsection{Veracity of Sources}
Judging the accuracy of the information and deciding relevant information content in social media references for the purpose of extracting entity identity attributes constitutes another challenging situation. For trending topics the search engines have started showing real-time feeds from social media websites in their search results. This has attracted spammers who post trending hash-tags or keywords along with their spam content in order to attract people to their websites offering products or services \cite{benevenuto2010detecting}. An alarming 355\% growth of social spam has been reported in 2013\footnote{http://www.likeable.com/blog/2013/11/10-surprising-social-media-statistics/}. Social media has also been instrumental in spreading misinformation and rumors. Spread of misinformation not only results in pandemonium among the users\footnote{http://www.theguardian.com/uk/interactive/2011/dec/07/london-riots-twitter}  but also result in extraction of completely wrong information about entities.

\subsection{Informal Text}
Unlike sources of news media and edited documents on the web, the textual content of the social media sources are highly colloquial and pose great difficulties in extracting information. One of the most important sources of information about events, prevalent in the domain of social media are the micro-blogging platforms. Micro blogs pose additional challenges due to their brevity, noisiness, idiosyncratic language, unusual structure and ambiguous representation of discourse \cite{bontcheva2013twitie}. Variation in language, less grammatical structure of sentences, unconventional uses of capitalization, frequent use of emoticons, and abbreviations have to be dealt by any system processing social media content. Moreover, various signals of communications embedded in the text in the form of hash-tags (eg.\#sochi), retweets (RT) and user mentions (@) should be understood by the system in order to extract the contextual information hidden in the text. Intentional misspellings sometimes demonstrate examples of intonation in written text \cite{prevost1996information}. For instance, expressions like, `this is so cooool', emphasizes stress on the emotions and conveys more information that should be captured. It has been shown that it is extremely challenging for the state-of-the art information extraction algorithms to perform efficiently and give accurate results for micro-blogs \cite{derczynski2013microblog}. For example, named entity recognition methods typically show 85-90\% accuracy on longer texts, but 30-50\% on tweets \cite{ritter2011named}. Status messages in social networking websites, content in question answering websites, reviews, and discussions in blogs, and forums exhibit similar nature and present similar challenges to information extraction and text mining procedures.



\subsection{Sampling Bias}
Most commonly used method for obtaining data samples from social media websites is by using their application programming interfaces (APIs). Given the humungous amounts of data produced in real-time, the APIs cannot provide all the data to every single API requests. The requests are often made through a query interface by passing certain query parameters to the APIs. The amount of data returned against the queries may vary. This depends upon the popularity of the content related to the query. For example, in Twitter studies have estimated that by using Twitter's Streaming API users can expect to receive anywhere from 1\% of the tweets to over 40\% of tweets in near real-time\footnote{https://www.brightplanet.com/2013/06/twitter-firehose-vs-twitter-api-whats-the-difference-and-why-should-you-care/}. The only way to get access to all the tweets is to buy the firehose service, which is seldom done for academic purposes. Other real-time social media publishing services mostly follow the same model. Therefore, this might lead to biasness in the samples collected for studying event related phenomenon and for tracking all the important event related information being produced in real-time.

\subsection{Multiple Data Sources}
The APIs (Application Programming Interfaces) of the different social media websites returns data in different formats (JSON, XML) using different web standards (REST, HTTPS). Moreover, the information obtained from a social media website is dependent upon the type of content it produces. A video sharing website might return an entirely different set of information from a blogging website. Thus, integrating the data obtained from the various social media platforms for the purpose of extraction and tracking of event related information is also one of the challenges.

\subsection{Lack of Evaluation Datasets}
There is a lack of ground truth evaluation data for most of the social media text mining tasks. In traditional data mining research, there is often two types of datasets. One of them is known as training dataset and the other is known as test dataset. The models are trained or developed using the training datasets and are evaluated on test datasets. Thus, the test datasets act as the ground truth. The test dataset for various text mining tasks is mostly not available for social media data. It is often the duty of the researchers to create new test datasets in order to solve a specific task in social media. Sometimes this data might not be a benchmark dataset due to various unwanted noise and human error or perception in annotating the data. This might lead to wrong assumptions and false results.


\section{Research Questions}

\section{Research Methodology}

\section{Research Contributions}

\section{Structure of the Thesis}

