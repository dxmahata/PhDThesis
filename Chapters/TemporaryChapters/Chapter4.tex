% Chapter 4

\chapter{Identifying Event-specific Sources from the Blogosphere} % Main chapter title

\label{BlogStudy} % For referencing the chapter elsewhere, use \ref{Chapter1} 

\lhead{Chapter 4. \emph{Identifying Event-specific Sources from the Blogosphere}} % This is for the header on each page - perhaps a shortened title

\begin{abstract}
 Social media has become a useful medium for mobilizing support for various real-life events and a platform for the public to voice their opinion freely to a huge audience in the web. Social media sources often provide novel and specific information in contrast to the generic information obtained from the mainstream media. This makes social media a valuable source for conducting studies and analyzing events. However, due to the power law distribution of the Internet, these sources get buried in the Long Tail. The overwhelming number of Long Tail social media sources makes it more challenging to identify the valuable sources with specific information. It is, therefore, of utmost importance to identify quality sources from these social media sites for understanding and exploring an event. We propose an evolutionary mutual reinforcement model for identifying and ranking highly `specific' social media sources, otherwise buried in the Long Tail, and `close' entities related to an event. We also introduce a novel evaluation strategy, due to the absence of ground truth for validating the results. We observe a huge percentage gain in information between 25\% and 130\% against the baselines (viz., Google search and Icerocket blog search) from the sources ranked according to our model. Further, our ranking methodology is capable of identifying the highly informative sources much earlier than the widely-used baselines. Our model also shows its potential as an apparatus to analyze events at micro and macro scales. Data for the research is collected from various blogging platforms like blogspot, livejournal, wordpress, typepad, etc. and will be made publicly available for researchers.
\keywords{event analysis, social media, mutual reinforcement, specificity, closeness, information gain}
\end{abstract}


\section{Introduction}

Social media has brought a paradigm shift in the way people share information and communicate.  Social media played an important role in mobilizing events such as, `The Arab Spring', `Occupy Wall Street', `Sandy relief efforts', `London Riots',`The Spanish Revolution', among others. This led to a surge in citizen journalism all over the world, encouraging transnational participation. Thus, social media serves as a parallel, yet distinct source of information about real-life events along with the mainstream media space \cite{reese2007mapping}.

The mainstream media sources often gloss over the intricate details while covering a real-life event. They are often biased, regulated by the government, and may not portray the true picture of an event \cite{hamdy2012framing}. While, social media sources like blogs often contain unbiased, uninhibited, and unedited opinions from people. Blogs have been accepted as more credible sources of information over mainstream media sources by the weblog users \cite{johnson2004wag}. Thus the sources, which are obtained from social media could potentially provide a rather `closer'  or an ``on-ground" view of the events with novel information. The ``on-ground'" information gleaned from the social media affords opportunities to study various online social phenomenon from methodological and theoretical perspectives including, social movements, crowdsourcing, citizen journalism, collective behavior, collective action \cite{agarwal2011finding,agarwal2012online,agarwal2012raising}, and more.

\begin{figure}[htb]
\centering
\includegraphics[height=2in,width=4.5in]{Figures/Chapter3Figures/comp.pdf}
\caption{Top 10 entities from mainstream media and blogs.}
\label{fg:comp}
\end{figure}

\noindent \textbf{\textit{Motivation:}} An initial analysis of the top 10 entities obtained from the top 10 search results related to ``Egyptian Revolution'' from two mainstream media channels (BBC and CNN), and from blogs during the time of the revolution is shown in Fig~\ref{fg:comp}. The top entities from the mainstream media channels are generic and are quiet obvious for the event. In contrast, the top entities from the blogs are very specific to the event. The activists like `Mona Seif', `Sarah Carr', `Maikel Nabil' and `Hosam El Hamalwy' were very closely involved, and were responsible for mobilizing the event. The entities like `Internet Blackout' and `Khaleed Saeed' were central to the event. Moreover, the presence of entities like `Facebook' and `Ghonim' (who was responsible for spreading the event in Facebook) among the top mainstream media entities also indicates the significance of social media in the event.

\begin{figure}[htb]
\centering
\includegraphics[height=3.5in,width=4.5in]{Figures/Chapter3Figures/touchgraph.jpg} 

\caption{\small Top 10 Google search results for ``Egyptian Revolution'', ``Libyan Revolution'', and ``Tunisian Revolution'', visualized using TouchGraph.}

\label{fg:touchgraph}

\end{figure}

Due to the power law distribution of the Internet \cite{adamic2000power}, and the present search engine technology, the `Short Head' is generally dominated by the mainstream media websites. As illustrated in Fig ~\ref{fg:touchgraph} the top 10 search results for ``Egyptian Revolution", ``Libyan Revolution", and ``Tunisian Revolution" by Google, visualized using Touchgraph\footnote{http://touchgraph.com}, retrieved mainstream media sources. Consequently, the social media sites get buried in the Long Tail \cite{LOmariba} as shown in Fig ~\ref{fg:longvsshort}. However sources from the social media channels, act as hubs of specific information about real-life events \cite{harb2011arab}. Thus, a person interested to analyze an event may miss out the novel and specific information available in social media by relying on the top results from the popular search engines. Moreover, in the words of Chris Anderson \cite{anderson2008long}, \begin{itshape} \small ``With an estimated 15 million bloggers out there, the odds that a few will have something important and insightful to say are good and getting better.'' \end{itshape} This motivated us to look for techniques in this paper, that would help in identifying these otherwise buried sources providing highly specific information related to an event.

\paragraph
\noindent \textbf{\textit{Challenges:}} Identifying highly informative `specific' sources and `close' entities related to a real-life event from social media entails various challenges as follows,
\begin{itemize}
\item \textbf{Sparsity of sources:} Enormous population of the sparsely linked Long Tail social media sources %\cite{agarwal2008identifying}% makes it challenging to identify and collect quality sources. 


\item \textbf{Quality assessment dilemma:} The entities (person, organization, place, etc.) mentioned in the sources act as the atomic units of information. Sources which are `specific' to an event must contain entities `closer' or highly relevant to the event. On the other hand, such `close' entities can be obtained from the `specific' sources. This presents a dilemma in assessing the quality of the sources for event related `specific' information content, and makes it a nontrivial task. 

\item \textbf{Entity extraction:} It is also a challenge, to accurately extract the entities from the social media sources, which are mostly unstructured and have colloquial content. 

\item \textbf{Lack of evaluation measures:} Conventional information retrieval based evaluation measures help in identifying the most relevant and authoritative sources, however, these sources may not be the most novel or offer specific information. Therefore, new evaluation measures are required to estimate the performance of our work.

\end{itemize}


\begin{figure}[htb]
\centering
\includegraphics[height=3.3in,width=4.8in]{Figures/Chapter3Figures/LongTailVsShortHead.jpg} 

\caption{\small Short Head Vs Long Tail media sources.}

\label{fg:longvsshort}

\end{figure}

\paragraph
\noindent \textbf{\textit{Contributions:}} We make the following contributions,
\begin{itemize}
\item \textbf{Methodology:} A methodology based on the principle of mutual reinforcement, that helps in identifying highly `specific' sources and `close' entities from social media, and their relationships (Section \ref{mutual}). It ranks the sources and the entities based on their `specific' information content and how `close' they are with respect to a set of events.

\item \textbf{Evaluation Strategy:} Present the methodology, along with an objective evaluation strategy to validate our findings (Section \ref{basecomp}). 

\item \textbf{Experiment on sources related to real-life events:} Perform our experiments on sources and entities related to the events: `Egyptian Revolution', `Libyan Revolution', and `Tunisian Revolution' (Section \ref{exp}). However, the work is extendible to other types of events. 

\item \textbf{Event analysis:} Explore the utility of such a model in analyzing events (Section \ref{explore}) and conclude the work with future directions (Section \ref{future}).
\end{itemize}

Next, we present the related work and compare and contrast  these with the proposed approach, highlighting our contributions to the literature.

\section{\label{sec:related}\textbf{Related Work}}
In this section, we discuss about some of the previous research relevant to our work. We present studies on analyzing real-life events, identifying quality sources related to them and in general from the web.

User-generated data from various social media platforms, related to real-life events, have been studied to perform wide range of analysis. Socio-political inferences were drawn by studying sentiments and opinions of people towards public and political events from Twitter \cite{younus2011average}, as well as blogs \cite{singh2010mining}. Twitter has been extensively used as a source for analyzing information circulated during natural disasters and crisis situations \cite{vieweg2010microblogging,cheong2011social}. Tweets related to events have been extracted, summarized and visualized, in order to have a deeper understanding of the events \cite{marcus2011twitinfo,popescu2011extracting}. 

Our work is different from all such works, and would help in analyzing events from sources and entities, which are highly specific to an event along with the generic ones.

Due to huge number of informal sources in social media it is a difficult task to identify high quality sources related to the real-life events. Researchers have built semantic web models for efficient retrieval of event specific media sources \cite{troncy2010linking}. Event related contents have been found leveraging the tagging and location information associated with the photos shared in Flickr \cite{rattenbury2007towards}. \textit{Becker et al} \cite{becker2011selecting}, studied how to identify events and high quality sources related to them from Twitter. In order to identify the genuine sources of information, credibility and trustworthiness of event related information were studied from Twitter \cite{gupta2012evaluating}. New methods were investigated for filtering and assessing the verity of sources obtained from social media by journalists \cite{diakopoulos2012finding}. 

The work presented in this paper, finds quality sources related to an event from social media, in terms of the `specific' information content of the source, and is quiet different from all such works.

Several methods have been developed in the past for identifying and ranking quality sources from the web. PageRank \cite{brin1998anatomy} took advantage of the link structure of the web for ranking web pages. It was further improved for making it sensitive to topic based search \cite{haveliwala2003topic}. Graph based approaches were used for modeling documents and a set of documents as weighted text graphs, and for computing relative importance of textual units for Natural Language Processing \cite{erkan2004lexrank}. Mutual reinforcement principle was used for identifying Hubs and Authorities from a subset of web pages \cite{kleinberg1999authoritative}. 

Our work is distinct from all such works. The methodology of our work utilizes the relationship between sources and entities related to a real-life event. It then builds upon the principle of mutual reinforcement and modifies it to a evolutionary system for finding `specific' sources and `close' entities w.r.t an event (Subection \ref{mutual}). A comparative analysis with the conventional mutual reinforcement demonstrates a faster convergence and better performance of our evolutionary model (Subsection \ref{compconvevol}). Our framework also has the potential to be used as an apparatus for studying events in terms of the specific sources and close entities identified and ranked by our proposed methodology (Section \ref{explore}). Next, we present the formal problem definition.




\section{\label{problemdef}\textbf{Problem Definition}}

The number of sources related to an event in social media is overwhelming. All these sources may not provide useful information and needs to be processed in order to identify the valuable
sources providing specific information about the concerned event. Provided we have a set of events, a set of sources, and a set of entities related to each of these events, we need to rank these sources and entities from the most specific to the most generic ones, based on their information content.

\begin{figure}[htbp]
\includegraphics[height=3in,width=4.8in]{Figures/Chapter3Figures/frameworkIllustrated.jpg} 
\caption{\small Black box view of the problem.} 
\label{fg:problem}
\end{figure}

\paragraph
\noindent \textbf{Event:} \begin{itshape}We define an event to be a real-world incident, occurring at any place at any time or over a certain period of time.\end{itshape}

\paragraph
\noindent \textbf{Specificity and Closeness:} \begin{itshape}
\noindent Given a finite set of events $\xi$, we take an event $E_{j} \in \xi$ such that,  $1 \le j \le \mid \xi \mid$ , a set of `p' sources denoted by $\phi_{E_{j}}$, and a set of `q' entities denoted by $\sigma_{E_{j}}$, related to the event $E_{j}$. We define two functions $\kappa$ (\textit{specificity}) and $\tau$ (\textit{closeness}) such that:

\begin{equation}
\label{func1}
\kappa : S_{i} \rightarrow [0,1]
 \end{equation}

\begin{equation}
\label{func2}
\tau : e_{i} \rightarrow [0,1]
\end{equation}

\noindent where, $S_{i} (\in \phi_{E_{j}}$), is the $i^{th}$ source, and $e_{i} (\in \sigma_{E_{j}}$) is the $i^{th}$ entity, so that we can get two ordered sets ($\varphi_{E_{j}}$ and $\varsigma_{E_{j}}$) for the set of sources in $\phi_{E_{j}}$ and entities in $\sigma_{E_{j}}$, such that:

\begin{equation}
\varphi_{E_{j}} = \{S_{1},...,S_{i},S_{j},...,S_{p} \mid \kappa(S_{i}) \ge \kappa(S_{j}), i < j\}
\end{equation}

\begin{equation}
\varsigma_{E_{j}} = \{e_{1},...,e_{i},e_{j},...,e_{q} \mid \tau(e_{i}) \ge \tau(e_{j}), i < j\}
\end{equation}
\end{itshape}

\noindent $\varphi_{E_{j}}$ is ordered in decreasing order of how `specific' $S_{i}$ is w.r.t $E_{j}$. $\varsigma_{E_{j}}$ is ordered in decreasing order of how `close' $e_{i}$ is w.r.t $E_{j}$. 
A black-box view of the problem is shown in Fig ~\ref{fg:problem}.

\section{\label{method}\textbf{Methodology}}

\begin{figure*}[htb]
\includegraphics[height=3.8in,width=4.8in]{Figures/Chapter3Figures/entitySources.jpg} 
\caption{\small Entities associated with a social media source.} 
\label{fg:sourceEntities}
\end{figure*}


A real-life event is characterized by a distinct set of close entities (persons, places, organizations, etc.) along with generic ones. The entities act as the basic units of information in these sources as shown in Fig ~\ref{fg:sourceEntities}. Intuitively, specific sources would contain closer entities and one is likely to find closer entities in more specific sources. The relation between specific sources and close entities could then be modeled following the Mutual Reinforcement Principle, which forms the basis of our methodology.

%%%%%%%%%%%%%%%%%%%%%%%%%%%%%%%%%%%%%%%%%%%%%%%%%%%%%%%%%%%%%%%%%%%%
%\begin{quote}
%An entity $e_{i}$ should have high `closeness' score $(\tau(e_{i})_{E_{j}})$  w.r.t an event $E_{j}$ if it appears in many sources $(\in \varphi_{E_{j}})$ with high 'specificity' scores $(\kappa(S_{i})_{E_{j}}$ while a source should have a high 'specificity' score if it contains many entities with high 'closeness' scores.
%\end{quote}
%%%%%%%%%%%%%%%%%%%%%%%%%%%%%%%%%%%%%%%%%%%%%%%%%%%%%%%%%%%%%%%%%%%%

\begin{quote}
An entity should have high `closeness' score if it appears in many sources with high 'specificity' scores while a source should have a high 'specificity' score if it contains many entities with high 'closeness' scores.
\end{quote}

In essence the principle states that the 'closeness' score of an entity is determined by the 'specificity' scores of the sources it appears in, and the 'specificity' score of a source is determined by the 'closeness' scores of the entities it contain. The proposed methodology extends the basic Mutual Reinforcement Principle to consider the evolving knowledge learned about an event. However, the model requires an apriori or seed knowledge about an event, which is provided in terms of an event profile or an event dictionary. Next, we discuss the construction of event dictionaries.

\subsection{\label{eventdict}\textbf{Event Dictionaries}}
Each event $E_{j}$ is profiled by constructing an event dictionary ($\sigma_{E_{j}}$). In order to calculate specificity of a source w.r.t an event, we need to start with an initial set of close entities. At the same time, these close entities are better acquired from the specific sources. To solve this dilemma, we construct event dictionaries, from independent sources which are completely separate from the sources ($\phi_{E_{j}}$) that need to be ranked. 

\paragraph
\noindent \textbf{\textit{Formulation of initial closeness scores:}} We calculate the `closeness' score ($\tau(e_{i})_{E_{j}}$) of each entity ($e_{i}$) for event $E_{j}$  in order to construct the event dictionaries, by using equations \ref{tfidf1} and \ref{tfidf2} based on tf-idf  measure \cite{ramos2003using} , from the information retrieval literature. Let $E_{j} \in \xi$, be the $j^{th}$ event, and `$e_{i}$' be the $i^{th}$ entity extracted from the set of sources selected for constructing the event dictionaries. If the term $f(e_{i}, E_{j})$ denotes the frequency of occurrence of the entity `$e_{i}$' in the set of sources for the event $E_{j}$, and $IE_{j}f(e_{i})$ denotes the inverse event frequency for the entity `$e_{i}$' then closeness score ($\tau(e_{i})_{E_{j}}$) of an entity $e_{i}$ w.r.t the event $E_{j}$ is defined as,

\begin{equation}
\label{tfidf1}
\tau(e_{i})_{E_{j}} = e_{i}f\_IE_{j}f = f(e_{i}, E_{j}) * IE_{j}f(e_{i})
\end{equation}

\begin{equation}
\label{tfidf2}
IE_{j}f(e_{i}) = log(\frac{\mid \xi \mid}{\mid E_{j} \in \xi : e_{i} \in E_{j} \mid})
\end{equation}

\noindent and, $\mid E_{j} \in \xi : e_{i} \in E_{j} \mid$ refers to the number of events in which the entity $e_{i}$ occurs. Since we extract the entities from the sources related to the events, we cannot have an entity that does not belong to any of the events. Therefore, we always have $\mid E_{j} \in \xi : e_{i} \in E_{j} \mid \ > 0$.

We get $\mid \xi \mid$ number of event dictionaries, each corresponding to an event. Following steps are taken to construct the event dictionaries:
\begin{enumerate}
\item \textit{\textbf{Entity Extraction:}} Entities are extracted from all the sources collected from GlobalVoices\footnote{http://globalvoicesonline.org} as explained in Section \ref{data}, using AlchemyAPI\footnote{http://alchemyapi.com} and their corresponding $\tau(e_{i})_{E_{j}}$ values are calculated using equation \ref{tfidf1}. We choose GlobalVoices for obtaining the seed sources for constructing the initial event dictionaries, as it is a portal where bloggers and translators work together to make reports of various real-life events, from blogs and citizen media everywhere.This makes it a reliable source for finding specific information content from social media. Due to colloquial nature of the sources as discussed in the challenges, some of the entities occur in several forms. For example, the entity `Tahrir Square' occur as `Tahreer', `El-Tahrir', etc. We resolve such multiple representation of the same entity by applying pattern matching\footnote{http://docs.python.org/2/library/difflib.html}. Given two entities represented as strings we accept them to be the same if their patterns match by 80\% or more. We would like to use the standard entity resolution algorithms in our future work.


\item \textit{\textbf{Closeness Score Computation:}} For each event $E_{j}$, we calculate $\tau(e_{i})_{E_{j}}$ scores for the set of entities for that event using equations \ref{tfidf1} and \ref{tfidf2}. An entity may occur in multiple events and hence can be present in multiple event dictionaries with different $\tau(e_{i})_{E_{j}}$ scores.
\item \textit{\textbf{Ranking:}} The higher the $\tau(e_{i})_{E_{j}}$ score of an entity the closer it is to the event. The entities are then ranked according to the descending $\tau(e_{i})_{E_{j}}$ scores.
\item \textit{\textbf{Normalization:}} Since the range of closeness scores are different for each event, we normalize $\tau(e_{i})_{E_{j}}$ scores w.r.t an event between 0 and 1. The normalization enables an assessment of relative closeness of an entity across multiple events. 
%affords a relative closeness assessment of an entity across multiple events.
\end{enumerate}

The dictionaries thus obtained from the above mentioned procedure are static and serve as a good source of apriori knowledge about the event. However, as we discover new knowledge from specific sources, it is desirable to update the event dictionaries. However, the method applied for constructing the initial event dictionaries require a set of events. This is a drawback of the current method and we plan to improve it in a future work. Next, we discuss how the dictionaries help in identifying specific sources, which in turn help in improving the dictionary.

\subsection{\label{mutual}\textbf{Mutually Reinforcing Sources and Entities}}

Given an event $E_{j} \in \xi$, a set of sources ($\phi_{E_{j}}$) and entities ($\sigma_{E_{j}}$), related to the event, we define two column vectors: \textbf{\textit{`Specificity'}} ($\mathbf{\kappa_{E_{j}}}$) and \textbf{\textit{`Closeness'} ($\mathbf{\tau_{E_{j}}}$)}.

\begin{equation}
\mathbf{\kappa_{E_{j}}} = <\kappa(S_{1})_{E_{j}},\kappa(S_{2})_{E_{j}},...,\kappa(S_{p})_{E_{j}} >^{T}
\end{equation}

\begin{equation}
\mathbf{\tau_{E_{j}}} = <\tau(e_{1})_{E_{j}},\tau(e_{2})_{E_{j}},...,\tau(e_{q})_{E_{j}} >^{T}
\end{equation}

\noindent where, $\kappa(S_{i})_{E_{j}}$ ($\in range(\kappa)$, from equation \ref{func1}) represents the \textit{`specificity'} score of $i^{th}$ source $S_{i} (\in \phi_{E_{j}})$, for $1 \le i \le p$ and $\tau(e_{i})_{E_{j}}$ ($\in range(\tau)$, from equation \ref{func2}) represents the \textit{`closeness'} score of $i^{th}$ entity $e_{i} (\in \sigma_{E_{j}})$, for $1 \le i \le q$. Each source $S_{i}$ may contain related as well as unrelated information about various events. If we consider the set of events $\xi$, then each $\kappa(S_{i})_{E_{j}}$ is itself a vector of `specificity' values of the source $S_{i}$ w.r.t the events $(\in E_{j})$ as expressed in equation \ref{specificity}.

\begin{equation}
\label{specificity}
\kappa(S_{i})_{E_{j}} = <\kappa(S_{i})_{E_{1}},\kappa(S_{i})_{E_{2}},...,\kappa(S_{i})_{E_{\mid \xi \mid}}>
\end{equation}

\noindent Similarly, each entity  $e_{i}$ may be related to various events. If we consider the set of events $\xi$, then each $\tau(e_{i})_{E_{j}}$ is itself a vector of `closeness' values of the entity $e_{i}$ w.r.t the events $(\in E_{j})$ as expressed in equation \ref{closeness}.

\begin{equation}
\label{closeness}
\tau(e_{i})_{E_{j}} = <\tau(e_{i})_{E_{1}},\tau(e_{i})_{E_{2}},...,\tau(e_{i})_{E_{\mid \xi \mid}}>
\end{equation}

\noindent However, while representing $\kappa(S_{i})_{E_{j}}$ and $\tau(e_{i})_{E_{j}}$ as an element of the vectors $\mathbf{\kappa_{E_{j}}}$ and $\mathbf{\tau_{E_{j}}}$, respectively, we only choose the entry for the $j^{th}$ event under consideration.

\begin{figure*}[htb]
\center
\includegraphics[height=4in,width=2.8in]{Figures/Chapter3Figures/graphnew.jpg} 
\caption{\small Bipartite graph G representing the mutual relationship between the sources and the entities.} 
\label{fg:graph}
\end{figure*}

We construct a bipartite graph $G = (V, U)$ (Figure~\ref{fg:graph}) representing the mutual relationship between the sources and the entities, where $V \in \phi_{E_{j}}, \sigma_{E_{j}}$, is the set of vertices for $G$, and $U$ is the set of undirected edges. The sources without entities are discarded during this process.

The presence of an entity in a source is not sufficient to determine its specificity. In order to express the specificity of a source w.r.t an event, we need to consider the closeness value of the entities present in the source. Given the closeness $\tau(e_{n})_{E_{j}}$ of an entity $e_{n}$ w.r.t an event $E_{j}$, obtained from the event dictionary, a weight $w_{mn}$ is assigned to the edges of the graph, which expresses the magnitude by which an entity is related to a source $S_{m}$.  

The significance of an entity $e_{n}$ in a source $S_{m}$ is expressed as,
\begin{equation}
\frac{f(e_{n}, S_{m})}{\displaystyle\sum\limits_{n=0}^{q} {f(e_{n}, S_{m})} }
\end{equation}

\noindent where, $f(e_{n}, S_{m})$ is the frequency of occurrence of the entity $e_{n}$ in the source $S_{m}$. Therefore mathematically,
\begin{equation}
w_{mn} = \frac{\tau(e_{n})_{E_{j}}\ast f(e_{n}, S_{m}) }{\displaystyle\sum\limits_{n=0}^{q} f(e_{n}, S_{m}) }
\end{equation}

The adjacency matrix of the bipartite graph G is denoted by L, and is defined as follows:

\[
L_{mn} = \left\{ \begin{array}{ccc}
w_{mn} & ~~ & if\  (m,n)\in U \\\\
0 & ~~ & otherwise \end{array} \right.
\]

%\noindent where, \textit($w_{mn}$) is the edge weight connecting source m and entity n. Mathematically,
%
%\begin{equation}
%w_{mn} = \frac{\tau_{E_{je_{n}}}\ast e{f_{mn}}}{\displaystyle\sum\limits_{n=0}^{q} e{f_{mn}} }
%\end{equation}
%
%\noindent where, $\tau_{E_{je_{n}}}$ is the closeness score of the entity $e_{n}$, obtained from the event dictionary (Section \ref{eventdict}) for the event $E_{j}$. . $\tau_{E_{je_{n}}}$ measures the significance of the entity w.r.t the event $E_{j}$ and the portion $\frac$



%\framebox{


%\squeezeup
%\begin{figure}[htbp]
%\begin{center}
%
%\includegraphics[height=3in,width=1.5in]{Figures/graph.jpg} 
%\caption{\small Source-entity graph for event $E_{j}$.} 
%\label{fg:figure1}
%
%\end{center}
%\end{figure}

\begin{figure}[htb]
\begin{center}

\includegraphics[height=4in,width=4.5in]{Figures/Chapter3Figures/algo.pdf} 
\caption{\small Algorithm for calculating `specificity' and `closeness'.} 
\label{fg:algo}
\end{center}
\end{figure}




Following the Mutual Reinforcement Principle the relationships between specificity scores of sources and closeness scores of entities for event $E_{j}$ can be denoted as follows,
\begin{equation}
\label{eq3}
\mathbf{\kappa_{E_{j}}} = \mathbf{L}\mathbf{\tau_{E_{j}}} 
\end{equation}


\begin{equation}
\label{eq4}
\mathbf{\tau_{E_{j}}} = \mathbf{L^{T}}\mathbf{\kappa_{E_{j}}}
\end{equation}

\noindent Substituting the values for $\mathbf{\kappa_{E_{j}}}$ and $\mathbf{\tau_{E_{j}}}$, we derive the following equations,
\begin{equation}
\label{eq5}
\mathbf{\kappa_{E_{j}}} = \mathbf{LL^{T}}\mathbf{\kappa_{E_{j}}} 
\end{equation}

\begin{equation}
\label{eq6}
\mathbf{\tau_{E_{j}}} = \mathbf{L^{T}L}\mathbf{\tau_{E_{j}}}
\end{equation}

\noindent Equations \ref{eq5} and \ref{eq6} are characteristic equations of an eigensystem, where the solutions to $\mathbf{\kappa_{E_{j}}}$  and $\mathbf{\tau_{E_{j}}}$ are the respective eigen vectors with the corresponding eigenvalue of 1. 

To emphasize the relationship between the sources and the entities, we make a major contribution by modifying the way the equations \ref{eq5} and \ref{eq6} are solved. We make the matrices $\mathbf{LL^{T}}$ and $\mathbf{L^{T}L}$ evolutionary while solving the equations. Since each of the equations is a circular definition, the final specificity and closeness scores are computed using the power iteration method \cite{golub1996matrix}. Each iteration improves specificity and closeness scores reflecting their mutual relationship. As we move towards getting the specific sources and close entities in each iteration, we update the weights $(w_{mn})$ assigned to the edges between the sources and the entities by the newly calculated closeness scores for the entities. This results in renewed reinforcement of the relationship at every iteration  by getting closer entities from better sources and vice-versa. This essentially helps the model incorporate the newly discovered knowledge about the events. More precisely, the improved understanding of the relationship between the source and the entities vis-a-vis an event is incorporated into the model.



The updation of the edge weights and the matrices with $k^{th}$ iteration is represented as follows,

\begin{equation}
w_{mn(k)} = \frac{\tau(e_{n\mathbf{(k-1)}})_{E_{j}}\ast f(e_{n}, S_{m})}{\displaystyle\sum\limits_{n=0}^q f(e_{n}, S_{m})} 
\end{equation}


\[
L_{mn(k)} = \left\{ \begin{array}{ccc}
w_{mn(k)} & ~~ & if\  (m,n) \in U \\\\
0 & ~~ & otherwise \end{array} \right.
\]

\noindent where, $L_{mn(k)}$ represents the adjacency matrix for graph G, and $w_{mn(k)}$ denotes the edge weight for the edge between $m^{th}$ source and $n^{th}$ entity at the $k^{th}$ iteration. $\tau(e_{n\mathbf{(k-1)}})_{E_{j}}$ represents the closeness score of the entity $e_{n} $ w.r.t the event $E_{j} (\in \xi)$, obtained from the evolving event dictionary for event $E_{j}$ at $(k-1)^{th}$ iteration.

If, $\mathbf{\kappa_{E_{j}}(k)}$ and $\mathbf{\tau_{E_{j}}(k)}$ be the specificity and the closeness scores, at the $k^{th}$ iteration, the iterative process for generating the final solution are,

\begin{equation}
\label{eq1}
\mathbf{\kappa_{E_{j}}(k)} = \mathbf{L_{k-1}L^{T}_{k-1}}\mathbf{\kappa_{E_{j}}(k-1)} 
\end{equation}

\begin{equation}
\label{eq2}
\mathbf{\tau_{E_{j}}(k)} = \mathbf{L^{T}_{k-1}L_{k-1}}\mathbf{\tau_{E_{j}}(k-1)}
\end{equation}

In order to get 1 as the largest eigenvalue and, $\mathbf{\kappa_{E_{j}}} $ and $\mathbf{\tau_{E_{j}}}$ as the principal eigen vectors, the matrices $\mathbf{L_{k-1}L^{T}_{k-1}}$ and $\mathbf{L^{T}_{k-1}L_{k-1}}$ needs to be stochastic and irreducible \cite{langville2004deeper} at every step of our evolutionary process.  In the present case, since the graph G is a bipartite graph, matrices $\mathbf{L_{k-1}L^{T}_{k-1}}$ and $\mathbf{L^{T}_{k-1}L_{k-1}}$ are already irreducible.



%If the matrices $\mathbf{LL^{T}}$ and $\mathbf{L^{T}L}$ are stochastic and irreducible \cite{langville2004deeper} then 1 would be the largest eigenvalue and the  vectors $\mathbf{\kappa_{E_{j}}} $ and $\mathbf{\tau_{E_{j}}}$ would be the principal eigen vectors respectively. In the present case, since the graph G is a bipartite graph, matrices $\mathbf{\kappa_{E_{j}}} $ and $\mathbf{\tau_{E_{j}}}$ are already irreducible.




In order to make the matrices $\mathbf{L_{k-1}L^{T}_{k-1}}$ and $\mathbf{L^{T}_{k-1}L_{k-1}}$ stochastic, we take the following steps at each iteration,


\begin{itemize}
  \item Dividing the non-zero entries of the matrices  $\mathbf{L_{k-1}L^{T}_{k-1}}$ and $\mathbf{L^{T}_{k-1}L_{k-1}}$ by the summation of all the entries in a row. 


  \item Assigning $1/n$ to the zero entries of $\mathbf{L_{k-1}L^{T}_{k-1}}$ and $1/m$ to the zero entries of $\mathbf{L^{T}_{k-1}L_{k-1}}$, respectively.
\end{itemize}

\noindent The whole process is presented as an algorithm as shown in Fig \ref{fg:algo}.

\noindent 
We also perform our study using conventional binary static matrices represented as follows,


\[
L_{mn} = \left\{ \begin{array}{ccc}
1 & ~~ & if\  (m,n) \in U \\\\
0 & ~~ & otherwise \end{array} \right.
\]

The proposed evolutionary model outperforms the static model as validated by the results discussed in Section \ref{exp}.


%\begin{equation}
%\frac{\tau_{E_{je_{n\mathbf{(k)}}}}\ast e{f_{mn}}}{\displaystyle\sum\limits_{n=0}^q e{f_{mn}} } = \frac{\tau_{E_{je_{n\mathbf{(k-1)}}}}\ast e{f_{mn}}}{\displaystyle\sum\limits_{n=0}^q e{f_{mn}} } 
%\end{equation}

%\begin{equation}
%\mathbf{w_{(k)}} = \mathbf{w_{(k-1)}}
%\end{equation}







\section{\label{data} \textbf{Data Collection}}
\noindent \textbf{\textit{Motivation behind source selection:}} For many people blogs have become popular social media sources for satisfying interpersonal communication needs. Blogs act as a platform for masses to share their likes and dislikes, voice their opinions, provide suggestions and report news. Over the years blogging has matured from personal diaries to citizen journalistic sources providing live coverage of events beyond the professional newsrooms. Often mainstream media rely on blogs for reporting first-hand accounts of an event \cite{ekdale2007expression}. Other social media platforms like microblogs, social networks etc., also promote such activities. But, these platforms have very little scope to elaborately discuss about the events due to the limitations in the length of content allowed to be posted. However, these alternative platforms act as good sources for studying and tracking dissemination of information during real-life events. This motivated us to perform our experiments on sources collected from the blogging platforms instead of other social media websites.

\begin{table}[htbp]
\centering
\caption{Details of Data Collected.}
\label{tab:table1}
\begin{tabular}{|c|c|c| }
\hline
\textbf{Service Used} & \textbf{Event} & \textbf{Number of Blog Posts} \\ [0.5ex]
\hline
GlobalVoices & Egyptian Revolution & 234 \\
&Libyan Revolution & 86 \\
&Tunisian Revolution & 77\\
\hline 
Google Blogger & Egyptian Revolution & 579 \\
&Libyan Revolution & 600 \\
&Tunisian Revolution & 484 \\
\hline
Icerocket Blog Search & Egyptian Revolution & 5900 \\ 
&Libyan Revolution & 2198 \\
&Tunisian Revolution & 1220 \\
\hline
\end{tabular}
\end{table}

%The blog writers, also known as bloggers, loosely form their special interest communities where they debate and discuss issues, spread awareness, gather support, organize and mobilize campaigns - utilizing and in many ways demonstrating the democratic nature of the Internet. 


\paragraph
\noindent \textit{\textbf{Sources:}} Blog posts from GlobalVoices, Blogger\footnote{http://blogger.com} and Icerocket Blog Search\footnote{http://icerocket.com} respectively, are collected for the study. The details of the dataset used is given in Table ~\ref{tab:table1}. The dataset includes 11,378 blog posts from various blogging platforms like blogspot.com, wordpress.com, livejournal.com, typepad.com, etc. We also filter out the non-english blogs. The data from GlobalVoices is used for constructing event dictionaries ($\sigma_{E_{j}}$), as explained in Section \ref{method}. We collect blog posts related to the three events from Blogger using Google Search, and from other blogging platforms using Icerocket blog search. We perform our experiments on the sources ($\phi_{E_{j}}$) retrieved by the search engines due to the lack of ground truth and take the sources along with the ranks assigned to them by the search engines as our baseline (explained in Subsection \ref{basecomp}) The collected blog posts are parsed for extracting various information. However, we use the following information for our study: \textit{URL} of the blog and blogpost), \textit{blog text}, \textit{entities}, \textit{language}, and \textit{rank} of the post in the respective search engines used for collecting it. We use AlchemyAPI in order to extract entities. These datasets would be made available on request.













\section{\label{exp}\textbf{Experiment and Analysis}}



In this section, we describe the experiments performed. First, we discuss the experimental setup, followed by the comparative analysis between the proposed evolutionary mutual reinforcement model and conventional mutual reinforcement model. We then, introduce a novel evaluation strategy comparing the proposed model with two baseline models.
\subsection{\textbf{Experimental Setup}}
The methodology discussed earlier is implemented on the collected datasets. We take the following steps in order to perform the experiment,

\begin{itemize}
\item \textit{\textbf{Constructing the Event Dictionaries:}} We take $\xi$ = \{``Egyptian Revolution", ``Libyan Revolution", ``Tunisian Revolution"\} as our set of events. We construct the event dictionaries ($\sigma_{E_{j}}$) by using the sources from GlobalVoices (explained in Event Dictionary subsection). 

%\item The sources related to each of these events are collected from Google and Icerocket as explained in Data Collection section.

%The $\tau(e_{i})_{E_{j}}$ values of the entities are calculated. The distribution of $\tau(e_{i})_{E_{j}}$ scores of entities for the set of events is presented in Figure 6. The threshold values ($\alpha_{E_{j}}$) for each event $E_{j} \in \xi$ are decided after careful manual inspection for constructing the event specific-dictionaries.

\item \textit{\textbf{Implementing the Proposed Evolutionary Mutual Reinforcement Model:}} The algorithm, as presented in Figure \ref{fg:algo}, is implemented on the set of sources ($\phi_{E_{j}}$) from Blogger and Icerocket related to each event respectively, and the set of entities ($\sigma_{E_{j}}$) from the event dictionary corresponding to each event $E_{j}$. The threshold value for convergence $\mu$ is set to 1e-08.

%
% We perform our experiment on the sources retrieved by the search engines due to the lack of baselines (explained in the Baseline Comparisons subsection). 

\item \textit{\textbf{Obtaining Specific Sources and Close Entities:}} With the termination of the algorithm we get a ranked set of sources ($\varphi_{E_{j}}$) ordered in terms of their specific information content and entities ($\varsigma_{E_{j}}$) ordered in terms of how closely related they are to the event. 

\item \textit{\textbf{Conventional Mutual Reinforcement Model Approach:}} We also implement the conventional mutual reinforcement model on $\phi_{E_{j}}$ and $\sigma_{E_{j}}$ without considering the evolving matrices (explained in Subsection \ref{mutual}).

\end{itemize}



%We have $\xi$ = \{``Egyptian Revolution", ``Libyan Revolution", ``Tunisian Revolution"\} as our set of events under study. Algorithm \ref{fg:algo} is implemented on the set of sources related to each event, and the set of entities from the event dictionary constructed in Section \ref{eventdict}. The threshold value for convergence $\mu$ is set to a very small value of 1e-08. We also implement algorithm \ref{fg:algo} without considering the evolving matrices, as explained in Section.
%
%With the termination of the algorithm we get a ranked set of sources ordered in terms of its specific information content and entities ordered in terms of how closely it is related to the event, as explained in Section 4. We further analyze the ranking of the top K specific sources as identified by our methodology and observe the difference in their rankings as assigned by the search engines and as assigned by our model. Figure \ref{fg:figure22} shows ranks assigned by Google and Icerocket for the top 10 sources for each event, ranked according to our framework. We conclude, that  our framework could identify the sources, which often gets buried in the Long Tail and has the potential for presenting valuable information about the event.





%\begin{figure*}[htbp]
%\begin{center}$
%\begin{array}{ccc}
%\includegraphics[height=2in,width=2in]{Figures/EgyptCumulativeIcerocket.jpg} &
%\includegraphics[height=2in,width=2in]{Figures/LibyaCumulativeIcerocket.jpg} &
%\includegraphics[height=2.2in,width=2in]{Figures/TunisiaCumulativeIcerocket.jpg}
%\end{array}$
%\end{center}
%\caption{Validating`specificity' $(\kappa)$ for sources from Icerocket Blog Search.}
%\label{fg:figure4}
%\end{figure*}


%\begin{table}[htbp]
%
%\caption{\small Top 5 entities in the event specific and the event class dictionaries constructed for the set of events $\xi$.}
%\label{tab:table2}
%\centering
%\begin{tabular}{|l|p{9cm}|}
%\hline
%\textbf{Egyptian Revolution Specific Dictionary} & Tahrir Square, Egyptian government, Gigi Ibrahim, Alexandria, Wael Abbas. \\
%\hline
%\textbf{Libyan Revolution Specific Dictionary} & Tripoli, Muammar Al Gaddafi, North Atlantic Treaty Organization, Chad, United Kingdom \\ 
%\hline
%\textbf{Tunisian Revolution Specific Dictionary} & Tunisian government, Lin Ben Mhenni, Samir Feriani, Kasbah Square, RCD \\ 
%\hline
%\textbf{Socio-Political event dictionary} & Twitter, Iranian Government, Tear gas devices, Facebook, Big Social network \\ 
%\hline
%
%\end{tabular}
%\end{table}



\subsection{\label{compconvevol}\textbf{Comparing Conventional and Evolutionary Mutual Reinforcement Models}}


In order to compare the efficiency of the proposed evolutionary mutual reinforcement model with the conventional mutual reinforcement model we combine the sources collected for an event $E_{j}$ from Google search and Icerocket search. After that we run the proposed evolutionary mutual reinforcement model and conventional mutual reinforcement model on the combined set of sources and event dictionary ($\sigma_{E_{j}}$) for each event. The number of iterations taken by the power iteration method to converge in each case is analyzed in Figure \ref{fg:figure12}. It is observed that the number of iterations taken by the power iteration method is lesser, when we employ the evolutionary mutual reinforcement model. Hence, we observe a marked improvement in the performance of our evolutionary mutual reinforcement model over the conventional static mutual reinforcement model.

\begin{figure}[htbp]
\centering
\includegraphics[height=2.5in,width=3in]{Figures/Chapter3Figures/Iteration.pdf}

\caption{\small Comparison of number of iterations taken by the power iteration method to converge, for the set of sources related to events in $\xi$, with the proposed evolutionary mutual reinforcement model and the conventional static mutual reinforcement model.}
\label{fg:figure12}

\end{figure}

\noindent \textit{\textbf{Explanation for the improvement:}} The lesser number of iterations in the proposed evolutionary mutual reinforcement model can be explained by the introduction of the evolutionary weights assigned to the relationship between the sources and the entities. The improved closeness scores ($\tau(e_{i},E_{j}$)) at each iteration of the evolutionary model and the renewed reinforcement of the relationship between the entities and the sources decreases the number of iterations taken by the proposed evolutionary model to converge in comparison to the static conventional model. Next, we compare the performance of the proposed evolutionary model with the baselines and the conventional mutual reinforcement model in terms of how quickly the models help in identifying valuable information about the events.

\begin{figure*}[htb]
\centering$
\begin{array}{c}
\includegraphics[height=2in,width=3.5in]{Figures/Chapter3Figures/RankDistGoogle.jpg} \\
\includegraphics[height=2in,width=3.5in]{Figures/Chapter3Figures/RankDistIceRocket.jpg} 

\end{array}$
\caption{Rankings of the sources from Google Blogger and Icerocket based on `specificity' $(\kappa)$ values obtained from our model and the rankings assigned by Google Search and Icerocket Blog Search.}
\label{fg:figure22}
\end{figure*}

\begin{figure*}[htbp]
\centering$
\begin{array}{cc}
\includegraphics[height=2.3in,width=2.3in]{Figures/Chapter3Figures/GrayedImages/googleEgyptCumulative.jpg} &
\includegraphics[height=2.3in,width=2.5in]{Figures/Chapter3Figures/GrayedImages/icerocketEgyptCumulative.jpg} \\
\includegraphics[height=2.3in,width=2.3in]{Figures/Chapter3Figures/GrayedImages/googleLibyaCumulative.jpg} &
\includegraphics[height=2.3in,width=2.3in]{Figures/Chapter3Figures/GrayedImages/icerocketLibyaCumulative.jpg} \\
\includegraphics[height=2.3in,width=2.3in]{Figures/Chapter3Figures/GrayedImages/googleTunisiaCumulative.jpg} &
\includegraphics[height=2.3in,width=2.3in]{Figures/Chapter3Figures/GrayedImages/icerocketTunisiaCumulatice.jpg}
\end{array}$

\caption{Validating specific sources obtained from our model.}
\label{fg:figure3}
\end{figure*}






\subsection{\label{basecomp}\textbf{Baseline Comparisons}}

\noindent \textit{\textbf{Selecting the Baselines:}} Due to lack of benchmark datasets we use the search results obtained from Google and Icerocket Blog Search as baselines for validation. Standard information retrieval measures for evaluation (DCG, NDCG, MAP, MAP@10) could not be used due to the absence of ground truth. We also consider the sources ranked according to the conventional static mutual reinforcement model and show the effectiveness of our model in quickly gaining information about an event. We further analyze the ranking of the top K specific sources as identified by our methodology and observe the difference in their rankings as assigned by the search engines and as assigned by our model. Figure \ref{fg:figure22} shows ranks assigned by Google and Icerocket for the top 10 sources for each event, ranked according to our framework. We conclude, that  our framework could identify the sources that often gets buried in the Long Tail and has the potential for presenting valuable information about the event. Next, we propose a novel strategy to demonstrate the effectiveness of the specific sources ranked by our evolutionary mutual reinforcement model in identifying highly informative sources.


\noindent \textit{\textbf{Rationale Behind The Strategy:}} Search engines are designed to give the most relevant sources containing the close entities, related to a query for a given event. As these close entities have a very high probability to be associated with the event, we expect to gain valuable information about the event from these entities making the highly ranked sources very specific to the event. We use this notion to propose a novel evaluation strategy, showing that when sources related to an event are ranked according to our model, they provide more valuable information than the ranking order given by the search engines and the conventional mutual reinforcement model for the same set of sources.


\noindent \textit{\textbf{Implementation of The Strategy:}} We compare closeness ($\tau(e_{i})_{E_{j}}$) values of the entities obtained from the sources ranked according to the search engines, the conventional model  and our model respectively. Following steps are taken,
\begin{itemize}

\item \noindent \textit{\textbf{Preparing the Ranked Lists of Entities:}} From the three differently ranked lists (search engine, our evolutionary model and conventional static model), each source is visited and the entities are extracted from them. The `$\tau(e_{i})_{E_{j}}$' values are assigned to these entities by referring to the respective final event dictionary ($\varsigma_{E_{j}}$) and they are ranked in descending order. 

\item \noindent \textit{\textbf{Measuring the Information Gain:}} As we traverse the three list of sources, we obtain three list of same entities, arranged in different orders depending upon the ranking of the sources. In order to show the comparison  in the gain of information from the three lists we take the top `K' entities from each list and calculate the sum of their `$\tau(e_{i})_{E_{j}}$' values and plot them against the value of `K' in Figure \ref{fg:figure3}. We start from K=1 and go on increasing its value till the number of entities are exhausted in all the three lists. It is evident from Figures \ref{fg:figure3}, that the curves based on specificity ($\kappa(S_{i})_{E_{j}})$ quickly gains over the curves based on the search engines and the conventional model. 

%\vspace{-3pt}
\item \noindent \textit{\textbf{Analysis:}} Figure \ref{fg:figure3} shows that information about the event is gained quicker using our model, which could identify specific sources earlier than the search engines as well as the conventional model. We measure the maximum percentage gain of information in each set of sources related to the three events. There is a maximum gain of information (130.4\%) in case of sources obtained from Icerocket search related to `Tunisian Revolution', and a minimum gain of information (25.97\%) in case of the sources obtained from Icerocket search related to `Egyptian Revolution'. Also when the sources are ranked according to our methodology they gain the maximum percentage of information at the $9^{th}$ source in case of sources obtained from Google related to `Libyan Revolution'. This in turn implies that the sources ranked higher by our model are more specific than the ones ranked by the search engines and the conventional model. These highly specific sources are also very informative. When presented earlier they also help in learning useful information about the event due to the presence of close entities in them. As we already observed earlier that these sources are often Long Tail sources, we can conclude that Long Tail sources that are ranked lower by the search engines, when identified are often more specific than the highly ranked short head sources. 

\end{itemize}



%Apart from giving better results, our evolutionary model also showed faster convergence than the conventional static mutual reinforcement model. The number of iterations taken by the power iteration method to converge in each case is analyzed in Figure \ref{fg:figure12}. It is observed that the number of iterations taken by the power iteration method is lesser, whenever we employ our evolutionary mutual reinforcement model. 
\begin{table*}[htbp]

\caption{\small Top 5 entities in the event specific and the event class dictionaries constructed for the set of events $\xi$.}
\label{tab:table2}
\centering
\begin{tabular}{|l|p{5cm}|}
\hline
\textbf{Egyptian Revolution Specific Dictionary} & Tahrir Square, Egyptian government, Gigi Ibrahim, Alexandria, Wael Abbas. \\
\hline
\textbf{Libyan Revolution Specific Dictionary} & Tripoli, Muammar Al Gaddafi, North Atlantic Treaty Organization, Chad, United Kingdom \\ 
\hline
\textbf{Tunisian Revolution Specific Dictionary} & Tunisian government, Lin Ben Mhenni, Samir Feriani, Kasbah Square, RCD \\ 
\hline
\textbf{Socio-Political event dictionary} & Twitter, Iranian Government, Tear gas devices, Facebook, Big Social network \\ 
\hline
\end{tabular}
\end{table*}
\vspace{-3pt}

\section{\label{explore}\textbf{Further Exploration}}

\begin{figure}[htbp]
\centering$
\begin{array}{c}
\includegraphics[height=3in,width=4in]{Figures/Chapter3Figures/confusionMatrixEgypt.pdf} \\
\includegraphics[height=3in,width=4in]{Figures/Chapter3Figures/confusionMatrixLibya.pdf} \\
\includegraphics[height=3in,width=4in]{Figures/Chapter3Figures/confusionMatrixTunisia.pdf}


\end{array}$

\caption{Different categories of entities for the events.}
\label{fg:figure5}
\end{figure}

%The proposed model helps in identifying closer entities and specific sources related to an event. 

We use the proposed framework developed by us as an apparatus to further explore event characteristics and show its utility in analyzing events. We show the potential of the final event dictionaries ($\varsigma_{E_{j}}$) obtained for each event $E_{j}$ for gaining valuable information about the event and to identify the generic entities related to the class of events. 

The entities in the final event dictionaries ($\varsigma_{E_{j}}$) are examined further. Based on their frequency $f(e_{i},E_{j})$ and closeness $(\tau(e_{i})_{E_{j}})$ scores, we categorize the entities into the following categories: a. \textit{Close and Frequent}, b. \textit{Close but Not Frequent}, c. \textit{Not Close but Frequent}, and d. \textit{Neither Close nor Frequent} as shown in Figure 7. We categorize an entity in each event dictionary as `frequent' if it occurs more than a threshold value $\eta$ i.e $f(e_{i},E_{j}) \ge \eta$, and `close' if it has   a closeness score more than a threshold value $\alpha$ i.e $\alpha \ge \tau(e_{i},E_{j})$. After careful manual inspection we decided the values of $\eta$ and $\alpha$ to be 5 and 0.5 respectively. The values of $\eta$ and $\alpha$ is same for all the events $E_{j} \in \xi$. However, different thresholds could be examined in the future though. Each entry in a matrix, consists of top 10 entities that satisfy the thresholds for the corresponding row and column category. We further examine  these entities in the context of each of these events and report interesting observations. We present the observations for ``Egyptian Revolution'', next, however similar observations were made for other events.



%The classification of entities helped us to analyze the event dictionaries from different perspectives as explained in the following subsections.
\subsection{\textbf{Event-specific Popular and Close Entities}}
We analyze the entities in the \textit{Close and Frequent}, and \textit{Close but Not Frequent} categories. We find that \textit{Close and Frequent} entities are not only frequent but are also closely related to the event. In other words these are very popular entities related to the event. For example, the occurrence of `Tahrir Square' and `Egyptian government' in this category, is inevitable, as Tahrir Square is the place where the protest started against the Egyptian Government. These are also the entities that anyone comes to know about from the top search results given by the search engines as well as from mainstream media sources. 

%So we further classify them as \textit{event-specific popular} entities. 

On the other hand, \textit{Close but Not Frequent} category primarily consists of entities that add novel and useful insights to the events. The occurrence of `Internet Cafe' in this category clearly shows how the local people in Egypt used Internet Cafes for accessing various social media websites in order to coordinate and participate in the revolution. It is also the place where `Khaled Said', a 28 year old computer-wiz was arrested by the Egyptian police. He was brutally tortured to death, triggering the anger among the people of Egypt for a mass uprising against the dictatorial government.  January 25, 2011 is also known as `The Day of Anger' in the Egyptian Revolution, as it is the day that marked the start of a series of protests and riots in Egypt. So the occurrence of the entity, `Day of Anger' in this category is reasonable. 

%So we further classify them as \textit{event-specific local} entities. 

Drawing upon the differences between the two categories our findings suggests that although the category of \textit{Close and Frequent} entities give a lot of information about an event, it is also necessary to know about the entities of \textit{Close but Not Frequent} category. The entities belonging to \textit{Close but Not Frequent} category provides in-depth information about the event from the grass root level. The \textit{Close but Not Frequent} entities are likely to be found buried in the Long Tail sources. In order to gain maximum information about an event it is necessary to identify the entities from both the categories. They point to the key persons, places and organizations related to the event. In other words, these entities when present in a source makes it highly specific to the event. Both these categories of entities can prove to be vital sources of information about the event. 



\subsection{\textbf{Event-specific and Event-class specific dictionaries}}
Based on the categorization and the analysis conducted in the previous subsection we divide the final event dictionaries ($\varsigma_{E_{j}}$) for each event into \textit{event-specific} and \textit{event-class specific} dictionaries. The \textit{event-specific} dictionaries are comprised of the entities categorized as \textit{Close and Frequent}, and \textit{Close but Not Frequent} for each event $E_{j} \in \xi$ respectively. Whereas, the \textit{event-class specific} dictionary is comprised of the entities found in the \textit{Not Close but Frequent} and \textit{Neither Close nor Frequent} categories for all the events $E_{j} \in \xi$.

Table \ref{tab:table2} shows the top five entities in the \textit{event specific} dictionaries for each event $E_{j}$ and a socio-political event dictionary for the set of events $\xi$ under study. The entities in the \textit{event specific} dictionaries, when present in a source makes it highly specific to that particular event and contributes in gaining information about it. These entities can also help in conducting micro-analysis of an event by identifying precise details about the event. On the other hand, the entities of the \textit{event-class specific} dictionary provides shallow information about a specific event, and are useful in learning about a category of events, in this case, socio-political uprisings in the middle east. These entities can help in conducting macro-analysis of an event by classifying the event into a certain category (like socio-political, crisis, entertainment, economic, etc.) and point out the general characteristics of the event. However, the analysis requires a set of known events, which could be a perceiveable constraint. We plan to address this in the future work. 

%Next, we present the related work and compare and contrast  these with the proposed approach, highlighting our contributions to the literature.

%Thus we conclude that, the first two categories of entities are very helpful in conducting a micro analysis and in knowing precise details about the events. When present in a source they make it highly ‘specific’. The other two categories are helpful for macro analysis. Although they provide shallow information about a specific event, they are useful in learning about a category of events, in this case, socio-political uprisings.



%\subsection{Analyzing the dictionary}



%The event dictionaries finally obtained enable us to further conduct micro (related to very specific) and macro (related to generic) analysis of the events. The entities in the event dictionaries are examined further. Based on their frequency and closeness $(\tau(e_{i})_{E_{j}})$ values, we create confusion matrices of entities for each of the events as shown in Figure 9., categorizing the entities into the following categories: a. Close and Frequent, b. Close but Not Frequent, c. Not Close but Frequent, and d. Neither Close nor Frequent. After studying the frequency distribution and `$\tau$' values of the entities in the event dictionaries we decided to classify an entity as `frequent' if it occurs more than five times for the set of sources for that event and `close' if it has a value greater than equal to 0.5 for `$\tau$'.  Each entry in a matrix, consists of top 10 entities that satisfy the conditions for the corresponding row and column category. We further examine  these entities in the context of each of these events and make some interesting observations. We present the observations for ``Egyptian Revolution'', next, however similar observations could be made for other events.

%\begin{itemize}
%\item `\textit{Close and Frequent}' entities are not only popular but are also closely related to the event. For example, the occurrence of `Tahrir Square' and `Egyptian government' in this category, is inevitable, as Tahrir Square is the place where the protest started against the Egyptian Government. 
%
%%\item `\textit{Close but Not Frequent}' category primarily consists of `local' entities. These entities add useful insights to the events. The occurrence of `Internet Cafe' in this category clearly shows how the local people in Egypt used Internet Cafes for accessing various social media websites in order to coordinate and participate the revolution. It is also the place where Khaled Said, a 28 year old computer-wiz was arrested by the Egyptian police. He was brutally tortured to death, triggering the anger among the people of Egypt for a mass uprising against the dictatorial government.  January 25, 2011 is also known as `The Day of Anger' in the Egyptian Revolution, as it is the day that marked the start of a series of protests and riots in Egypt. So the occurrence of the entity, `Day of Anger' in this category is reasonable.
%
%\item `\textit{Not Close but Frequent}' and `\textit{Neither Close nor Frequent}' entities do not contribute much in gaining `specific' information about the events. Entities like `Facebook', `United States of America' , `New York Times', etc are popular in their own rights. They may not be important in the context of the indiividual events and provides very generic information about the set of events under analysis.
%
%\end{itemize} 
%Thus we conclude that, the first two categories of entities are very helpful in conducting a micro analysis and in knowing precise details about the events. When present in a source they make it highly ‘specific’. The other two categories are helpful for macro analysis. Although they provide shallow information about a specific event, they are useful in learning about a category of events, in this case, socio-political uprisings.

 



\section{\label{future}\textbf{Conclusions And Future Work}}

In this paper, we highlighted the need for exploring the social media sources to study an event that are often buried in the Long Tail. We demonstrated that social media sources have the capability to provide very specific and novel information. However, the sheer volume of social media sources and the Long Tail characteristics (e.g., link sparsity, colloquial language, etc.) make it extremely challenging to identify the specific sources. Towards this direction, we developed a methodology that utilizes relevant entities as a mechanism to identify specific sources in a mutual reinforcement framework. Further, in order to consider the dynamic relationship between the specific sources and close entities, an evolutionary mutual reinforcement model is developed. Experiments conducted on real-world datasets for the three social movements during the Arab Spring, viz., Egyptian Revolution, Libyan Revolution, and Tunisian Revolution demonstrate faster convergence and better accuracy of the evolutionary mutual reinforcement model over the conventional mutual reinforcement model. Furthermore, the evolutionary mutual reinforcement model outperformed one of the most-widely used search
engines, i.e., Google Blog Search and IceRocket. It was observed that the search engines ranked the specific sources surprisingly low, thereby reducing the chances of their discovery. The poor hyperlink connectivity of these Long Tail social media sources was contemplated to be a big reason behind their low ranks in traditional search engines. By analyzing the close entities identified by our model we also showed the potential of the framework to be utilized for analyzing events. In future, we plan to explore the proposed framework in assessing credibility of the Long Tail social media sources, which are known to disseminate false reports for breaking events. We also plan to use the framework in order to identify users with sustained interest in different events.
