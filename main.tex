%%%%%%%%%%%%%%%%%%%%%%%%%%%%%%%%%%%%%%%%%
% Masters/Doctoral Thesis 
% LaTeX Template
% Version 1.43 (17/5/14)
%
% This template has been downloaded from:
% http://www.LaTeXTemplates.com
%
% Original authors:
% Steven Gunn 
% http://users.ecs.soton.ac.uk/srg/softwaretools/document/templates/
% and
% Sunil Patel
% http://www.sunilpatel.co.uk/thesis-template/
%
% License:
% CC BY-NC-SA 3.0 (http://creativecommons.org/licenses/by-nc-sa/3.0/)
%
% Note:
% Make sure to edit document variables in the Thesis.cls file
%
%%%%%%%%%%%%%%%%%%%%%%%%%%%%%%%%%%%%%%%%%

%----------------------------------------------------------------------------------------
%	PACKAGES AND OTHER DOCUMENT CONFIGURATIONS
%----------------------------------------------------------------------------------------

\documentclass[11pt, oneside]{Thesis} % The default font size and one-sided printing (no margin offsets)

\graphicspath{{Pictures/}} % Specifies the directory where pictures are stored

\usepackage[square, numbers, comma, sort&compress]{natbib} % Use the natbib reference package - read up on this to edit the reference style; if you want text (e.g. Smith et al., 2012) for the in-text references (instead of numbers), remove 'numbers' 
\hypersetup{urlcolor=blue, colorlinks=true} % Colors hyperlinks in blue - change to black if annoying
\title{\ttitle} % Defines the thesis title - don't touch this

\begin{document}

\frontmatter % Use roman page numbering style (i, ii, iii, iv...) for the pre-content pages

\setstretch{1.3} % Line spacing of 1.3

% Define the page headers using the FancyHdr package and set up for one-sided printing
\fancyhead{} % Clears all page headers and footers
\rhead{\thepage} % Sets the right side header to show the page number
\lhead{} % Clears the left side page header

\pagestyle{fancy} % Finally, use the "fancy" page style to implement the FancyHdr headers

\newcommand{\HRule}{\rule{\linewidth}{0.5mm}} % New command to make the lines in the title page

% PDF meta-data
\hypersetup{pdftitle={\ttitle}}
\hypersetup{pdfsubject=\subjectname}
\hypersetup{pdfauthor=\authornames}
\hypersetup{pdfkeywords=\keywordnames}

%----------------------------------------------------------------------------------------
%	TITLE PAGE
%----------------------------------------------------------------------------------------

\begin{titlepage}
\begin{center}

\textsc{\LARGE University of Arkansas at Little Rock}\\[1.5cm] % University name
\textsc{\Large Doctoral Thesis}\\[0.5cm] % Thesis type

\HRule \\[0.4cm] % Horizontal line
{\huge \bfseries A Framework for Collecting, Extracting and Managing Event Identity Information from Textual Content in Social Media}\\[0.4cm] % Thesis title
\HRule \\[1.5cm] % Horizontal line
 
\begin{minipage}{0.4\textwidth}
\begin{flushleft} \large
\emph{Author:}\\
\href{https://sites.google.com/a/ualr.edu/debanjan-mahata/home}{Debanjan Mahata} % Author name - remove the \href bracket to remove the link
\end{flushleft}
\end{minipage}
\begin{minipage}{0.4\textwidth}
\begin{flushright} \large
\emph{Supervisor:} \\
\href{http://ualr.edu/technologyinnovation/faculty/john-talburt/}{Dr. John R. Talburt} % Supervisor name - remove the \href bracket to remove the link  
\end{flushright}
\end{minipage}\\[3cm]
 
\large \textit{A thesis submitted in fulfilment of the requirements\\ for the degree of \degreename}\\[0.3cm] % University requirement text
\textit{in}\\[0.4cm]
Integrated Computing \\ Information Quality Track \\ Department of Information Science \\[2cm] % Research group name and department name
 
{\large \today}\\[4cm] % Date
%\includegraphics{Logo} % University/department logo - uncomment to place it
 
\vfill
\end{center}

\end{titlepage}

%----------------------------------------------------------------------------------------
%	DECLARATION PAGE
%	Your institution may give you a different text to place here
%----------------------------------------------------------------------------------------

\Declaration{

\addtocontents{toc}{\vspace{1em}} % Add a gap in the Contents, for aesthetics

I, Debanjan Mahata, declare that this thesis titled, '\ttitle' and the work presented in it are my own. I confirm that:

\begin{itemize} 
\item[\tiny{$\blacksquare$}] This work was done wholly or mainly while in candidature for a research degree at this University.
\item[\tiny{$\blacksquare$}] Where any part of this thesis has previously been submitted for a degree or any other qualification at this University or any other institution, this has been clearly stated.
\item[\tiny{$\blacksquare$}] Where I have consulted the published work of others, this is always clearly attributed.
\item[\tiny{$\blacksquare$}] Where I have quoted from the work of others, the source is always given. With the exception of such quotations, this thesis is entirely my own work.
\item[\tiny{$\blacksquare$}] I have acknowledged all main sources of help.
\item[\tiny{$\blacksquare$}] Where the thesis is based on work done by myself jointly with others, I have made clear exactly what was done by others and what I have contributed myself.\\
\end{itemize}
 
Signed:\\
\rule[1em]{25em}{0.5pt} % This prints a line for the signature
 
Date:\\
\rule[1em]{25em}{0.5pt} % This prints a line to write the date
}

\clearpage % Start a new page

%----------------------------------------------------------------------------------------
%	QUOTATION PAGE
%----------------------------------------------------------------------------------------

\pagestyle{empty} % No headers or footers for the following pages

\null\vfill % Add some space to move the quote down the page a bit

\textit{``Torture the data, and it will confess to anything."}

\begin{flushright}
Ronald Coase, Economics, Nobel Prize Laureate
\end{flushright}

\vfill\vfill\vfill\vfill\vfill\vfill\null % Add some space at the bottom to position the quote just right

\clearpage % Start a new page

%----------------------------------------------------------------------------------------
%	ABSTRACT PAGE
%----------------------------------------------------------------------------------------

\addtotoc{Abstract} % Add the "Abstract" page entry to the Contents

\abstract{
%\addtocontents{toc}{\vspace{1em}} % Add a gap in the Contents, for aesthetics

With the popularity of social media platforms like Facebook, Twitter, Google Plus, etc, there has been voluminous growth in the digital footprints of real-life events in the Internet. The user generated colloquial and concise textual content related to different types of real-life events, produced in these websites,  acts as a hotbed for researchers and organizations for extracting valuable and meaningful information. There has been significant improvement in natural language processing techniques for mining formal and long textual content often found in blogs and newspaper articles. But, it is still a challenging task to mine textual information from the social media channels producing terse, informal and noisy text with an unusual structure. For  an event of interest it is necessary to detect and store event-specific signals from the noisy social media channels that allows to distinctively identify that event among all others and characterizes it for drawing actionable insights. These event-specific cues also forms its identity in the unstructured domain of social media. This identity information when mined and analyzed in a timely manner has tremendous applications in the areas of real-life event analysis, opinion mining, reference tracking, recommendation engines, cyber security, event management, among others. Thus, there is a need of a generic framework that can collect short textual content related to real-life events, extract information from them and maintain the information persistently for performing data analytics tasks, and tracking newly produced content as an event evolves. The patent pending work presented in this thesis establishes the design and implementation of an extendable framework enabling collecting, extracting and persistently managing identity information of real-life events from short textual content produced in social media. Towards this objective a pipeline of data processing components going through repeated processing cycles - \textit{Event Identity Information Management Life Cyle} (EIIM) is proposed. A novel persistent graph data structure - \textit{EventIdentityInfoGraph} representing the identity information structure of an event is implemented that forms the core component of the EIIM cycle. Mutually reinforcing relationships between event-specific social media posts, hashtags, text units, URLs and users, forming the vertices of the graph and denoting \textit{event identity information units}, are defined and quantified. An iterative and scalable algorithm - \textit{EventIdentityInfoRank} is proposed that processes the vertices of the graph and ranks them  in terms of event-specific informativeness by leveraging the mutually reinforcing relationships. The ranked \textit{event identity information units} are further used in tracking new event related content and extracting valuable event-specific information. Different components of the framework are tested and validated for real-time event related content generated in social media. The work is concluded by discussing about its novel contributions, practical applications in various other domains and envisaging future directions.
}



\clearpage % Start a new page

%----------------------------------------------------------------------------------------
%	ACKNOWLEDGEMENTS
%----------------------------------------------------------------------------------------

\setstretch{1.3} % Reset the line-spacing to 1.3 for body text (if it has changed)

\acknowledgements{\addtocontents{toc}{\vspace{1em}} % Add a gap in the Contents, for aesthetics

I would like to express the deepest appreciation to my committee chair Dr. John R. Talburt, who has shown the attitude and the substance of a genius. He continiously and persuasively conveyed a spirit of adventure in regard to research and scholarship, and an excitement in regard to directing innovation towards practical problems. Without his supervision and constant support this dissertation would not have been possible.

   I would like to thank my committee members, Dr. Elizabeth Pierce, Dr. Ningning Wu, Dr. Russel Bruhn and Dr. Mathias Brochhausen, whose high quality contributions in the field of Information Science and Information Quality have inspired me to set high standards in my work, and kept me motivated. I would specially thank Dr. Mathias Brochhausen for devoting his valuable time for discussing about possible applications of ontologies in representing real-life events and the related information content in social media. I strongly consider it as one of the future directions of my research.

    In addition, I thank Dr. Vivek Kumar Singh and his team from Banaras Hindu University, India, for collaborating with me and helping me to execute the necessary evaluation tasks in an unbiased way, including manual annotations and feedback. I also acknowledge the support of Mr. Jeff Stinson and Ms. Glediana Rexha for financially supporting the major part of my PhD by allowing me to work as a Graduate Assistant at TechLaunch, University of Arkansas at Little Rock. 

I am extremely thankful to Dr. Abhijit Bhattacharyya (Associated Dean, Donaghey College of Engineering and Information Technology), for providing me with advise and encouragement from time to time. This acknowledgement page would be incomplete without thanking the immense support of my friends and family. I thank my parents, wife and friends (specially Pathikrit Bhattacharya, Subhashish Duttachowdhury and Meenakshisundaram Balasubramaniam) for not only their support but for their constant interest in my work and the discussions that I had with them. The conversations with them helped me to understand the information seeking behavior of various people from social media, with different perspectives.

Lastly, I thank University of Arkansas for providing me with the facilities, funds and a congenial environment for working towards my goal of PhD. I also acknowledge the Board Of Trustees Of The University Of Arkansas for filing a provisional patent of my work and encouraging me to pursue a path of innovation.


}
\clearpage % Start a new page

%----------------------------------------------------------------------------------------
%	LIST OF CONTENTS/FIGURES/TABLES PAGES
%----------------------------------------------------------------------------------------

\pagestyle{fancy} % The page style headers have been "empty" all this time, now use the "fancy" headers as defined before to bring them back

\lhead{\emph{Contents}} % Set the left side page header to "Contents"
\tableofcontents % Write out the Table of Contents

\lhead{\emph{List of Figures}} % Set the left side page header to "List of Figures"
\listoffigures % Write out the List of Figures

\lhead{\emph{List of Tables}} % Set the left side page header to "List of Tables"
\listoftables % Write out the List of Tables

%%----------------------------------------------------------------------------------------
%%	ABBREVIATIONS
%%----------------------------------------------------------------------------------------
%
%\clearpage % Start a new page
%
%\setstretch{1.5} % Set the line spacing to 1.5, this makes the following tables easier to read
%
%\lhead{\emph{Abbreviations}} % Set the left side page header to "Abbreviations"
%\listofsymbols{ll} % Include a list of Abbreviations (a table of two columns)
%{
%\textbf{LAH} & \textbf{L}ist \textbf{A}bbreviations \textbf{H}ere \\
%%\textbf{Acronym} & \textbf{W}hat (it) \textbf{S}tands \textbf{F}or \\
%}

%%----------------------------------------------------------------------------------------
%%	PHYSICAL CONSTANTS/OTHER DEFINITIONS
%%----------------------------------------------------------------------------------------
%
%\clearpage % Start a new page
%
%\lhead{\emph{Physical Constants}} % Set the left side page header to "Physical Constants"
%
%\listofconstants{lrcl} % Include a list of Physical Constants (a four column table)
%{
%Speed of Light & $c$ & $=$ & $2.997\ 924\ 58\times10^{8}\ \mbox{ms}^{-\mbox{s}}$ (exact)\\
%% Constant Name & Symbol & = & Constant Value (with units) \\
%}

%%----------------------------------------------------------------------------------------
%%	SYMBOLS
%%----------------------------------------------------------------------------------------
%
%\clearpage % Start a new page
%
%\lhead{\emph{Symbols}} % Set the left side page header to "Symbols"
%
%\listofnomenclature{lll} % Include a list of Symbols (a three column table)
%{
%$a$ & distance & m \\
%$P$ & power & W (Js$^{-1}$) \\
%% Symbol & Name & Unit \\
%
%& & \\ % Gap to separate the Roman symbols from the Greek
%
%$\omega$ & angular frequency & rads$^{-1}$ \\
%% Symbol & Name & Unit \\
%}

%----------------------------------------------------------------------------------------
%	DEDICATION
%----------------------------------------------------------------------------------------

\setstretch{1.3} % Return the line spacing back to 1.3

\pagestyle{empty} % Page style needs to be empty for this page

\dedicatory{Dedicated to my parents, wife and my entire family for their endless love, support and encouragement.} % Dedication text

\addtocontents{toc}{\vspace{2em}} % Add a gap in the Contents, for aesthetics

%----------------------------------------------------------------------------------------
%	DISSERTATION OVERVIEW
%----------------------------------------------------------------------------------------
\clearpage % Start a new page

\begin{center} \textbf{\Huge Dissertation Overview} \end{center}

 % This is for the header on each page - perhaps a shortened title

\begin{figure}[htbp]
  \caption{Event Identity Information Management (EIIM) Life Cycle for user generated short textual content in social media}
  \centering
    \includegraphics[width=15.5cm,height=8cm]{Figures/EIIMComponents/fullFramework.jpg}
\end{figure}

\textbf{\LARGE Related Filed Patent}
\begin{itemize}
\item A System for Collecting, Ranking and Managing Entity Identity Information from Social Media (US 62135258). Inventors: \textbf{Debanjan Mahata} and John R. Talburt, Assignee: The Board Of Trustees Of The University Of Arkansas.
\end{itemize}

\textbf{\LARGE Related Publications}
\begin{itemize}
\item \textbf{Debanjan Mahata}, John R. Talburt and Vivek Kumar Singh; \textit{Identifying and Ranking of Event-specific Entity-centric Informative Content from Twitter}. $20^{th}$ International Conference On Applications Of Natural Language To Information Systems (NLDB 2015), Passau, Germany. $17^{th}-19^{th}$ June, 2015.

\item \textbf{Debanjan Mahata} and John R. Talburt; \textit{A Framework for Collecting and Managing Entity Identity Information from Social Media}. $19^{th}$ International Conference on Information Quality, Xi'An, China.

\item \textbf{Debanjan Mahata} and Nitin Agarwal; \textit{Identifying Event-specific Sources from Social Media}. Online Social Media Analysis and Visualization. Lecture Notes in Social Networks, Springer, Kawash, Jalal (Ed). January, 2015.

\item Nitin Agarwal, \textbf{Debanjan Mahata}, and Huan Liu. \textit{Time-and Event-Driven Modeling of Blogger Influence}. Encyclopedia of Social Network Analysis and Mining. Springer New York, 2014. 2154-2165.


\item \textbf{Debanjan Mahata} and Nitin Agarwal. \textit{Learning from the crowd: An Evolutionary Mutual Reinforcement Model for Analyzing Events}. Advances in Social Networks Analysis and Mining (ASONAM), 2013 IEEE/ACM International Conference on. IEEE, 2013.

\item Nitin Agarwal, and \textbf{Debanjan Mahata}. \textit{Grouping the Similar among the Disconnected Bloggers}. Social Media Mining and Social Network Analysis: Emerging Research (2013), 54.

\item \textbf{Debanjan Mahata}, and Nitin Agarwal. \textit{What does everybody know? identifying event-specific sources from social media}. IEEE Fourth International Conference on Computational Aspects of Social Networks (CASoN), 2012.

\item \textbf{Debanjan Mahata} and Nitin Agarwal. \textit{Analyzing Event-specific Socio-Technical Behaviors Through the Lens of Social Media}. The International Sunbelt Social Network Conference (Sunbelt XXXII) organized by the International Network for Social Network Analysis (INSNA), March 12-18, 2012, Redondo Beach, California.

\item Vivek Kumar Singh, \textbf{Debanjan Mahata}, and Rakesh Adhikari. \textit{Mining the blogosphere from a socio-political perspective}. IEEE International Conference on Computer Information Systems and Industrial Management Applications (CISIM), 2010.

\item Vivek Kumar Singh, Rakesh Adhikari, and \textbf{Debanjan Mahata}. \textit{A clustering and opinion mining approach to socio-political analysis of the blogosphere}. IEEE International Conference on Computational Intelligence and Computing Research (ICCIC), 2010.

\end{itemize}

\textbf{\LARGE Related Submitted Publications}

\begin{itemize}
\item \textbf{Debanjan Mahata}, John R. Talburt, Vivek Kumar Singh and Rajesh Piryani; \textit{Chatter that Matter: A Framework for Identifying and Ranking Event-specific Informative Tweets}. $18^{th}$ International Conference on Text, Speech and Dialogue, Plzen, Czech Republic (Notification Due: May 10, 2015)

\item \textbf{Debanjan Mahata}, John R. Talburt and Vivek Kumar Singh; \textit{A Framework for Collecting, Extracting and Managing Event Identity Information from Twitter}. $20^{th}$ International Conference on Information Quality, M.I.T, Boston (Notification Due: April 30, 2015)

\item \textbf{Debanjan Mahata}, John R. Talburt and Vivek Kumar Singh; \textit{From Chirps to Whistles : Discovering Event-specific Informative Content from Twitter}. Proceedings of the $7^{th}$ Annual ACM Web Science Conference. ACM, 2015, Oxford, England (Notification Due: April 30, 2015)

\end{itemize}


%----------------------------------------------------------------------------------------
%	THESIS CONTENT - CHAPTERS
%----------------------------------------------------------------------------------------

\mainmatter % Begin numeric (1,2,3...) page numbering

\pagestyle{fancy} % Return the page headers back to the "fancy" style

% Include the chapters of the thesis as separate files from the Chapters folder
% Uncomment the lines as you write the chapters
%% Chapter 1

\chapter{Dissertation Overview} % Main chapter title

\label{overview} % For referencing the chapter elsewhere, use \ref{Chapter1} 

\lhead{\emph{Dissertation Overview}} % This is for the header on each page - perhaps a shortened title

\begin{figure}
  \caption{Event Identity Information Management (EIIM) Life Cycle for user generated short textual content in social media}
  \centering
    \includegraphics[width=15.5cm,height=8cm]{Figures/EIIMComponents/fullFramework.jpg}
\end{figure}

\textbf{\LARGE Related Publications}
\begin{itemize}
\item \textbf{Debanjan Mahata}, John R. Talburt and Vivek Kumar Singh; \textit{Identifying and Ranking of Event-specific Entity-centric Informative Content from Twitter}. $20^{th}$ International Conference On Applications Of Natural Language To Information Systems (NLDB 2015), Passau, Germany. $17^{th}-19^{th}$ June, 2015.

\item \textbf{Debanjan Mahata} and John R. Talburt; \textit{A Framework for Collecting and Managing Entity Identity Information from Social Media}. $19^{th}$ International Conference on Information Quality, Xi'An, China.

\item \textbf{Debanjan Mahata} and Nitin Agarwal; \textit{Identifying Event-specific Sources from Social Media}. Online Social Media Analysis and Visualization. Lecture Notes in Social Networks, Springer, Kawash, Jalal (Ed). January, 2015.

\item Nitin Agarwal, \textbf{Debanjan Mahata}, and Huan Liu. \textit{Time-and Event-Driven Modeling of Blogger Influence}. Encyclopedia of Social Network Analysis and Mining. Springer New York, 2014. 2154-2165.


\item \textbf{Debanjan Mahata} and Nitin Agarwal. \textit{Learning from the crowd: An Evolutionary Mutual Reinforcement Model for Analyzing Events}. Advances in Social Networks Analysis and Mining (ASONAM), 2013 IEEE/ACM International Conference on. IEEE, 2013.

\item Nitin Agarwal, and \textbf{Debanjan Mahata}. \textit{Grouping the Similar among the Disconnected Bloggers}. Social Media Mining and Social Network Analysis: Emerging Research (2013), 54.

\item \textbf{Debanjan Mahata}, and Nitin Agarwal. \textit{What does everybody know? identifying event-specific sources from social media}. IEEE Fourth International Conference on Computational Aspects of Social Networks (CASoN), 2012.

\item \textbf{Debanjan Mahata} and Nitin Agarwal. \textit{Analyzing Event-specific Socio-Technical Behaviors Through the Lens of Social Media}. The International Sunbelt Social Network Conference (Sunbelt XXXII) organized by the International Network for Social Network Analysis (INSNA), March 12-18, 2012, Redondo Beach, California.

\item Vivek Kumar Singh, \textbf{Debanjan Mahata}, and Rakesh Adhikari. \textit{Mining the blogosphere from a socio-political perspective}. IEEE International Conference on Computer Information Systems and Industrial Management Applications (CISIM), 2010.

\item Vivek Kumar Singh, Rakesh Adhikari, and \textbf{Debanjan Mahata}. \textit{A clustering and opinion mining approach to socio-political analysis of the blogosphere}. IEEE International Conference on Computational Intelligence and Computing Research (ICCIC), 2010.

\end{itemize}

\textbf{\LARGE Related Submitted Publications}

\begin{itemize}
\item \textbf{Debanjan Mahata}, John R. Talburt, Vivek Kumar Singh and Rajesh Piryani; \textit{Chatter that Matter: A Framework for Identifying and Ranking Event-specific Informative Tweets}. $18^{th}$ International Conference on Text, Speech and Dialogue, Plzen, Czech Republic (Notification Due: May 10, 2015)

\item \textbf{Debanjan Mahata}, John R. Talburt and Vivek Kumar Singh; \textit{A Framework for Collecting, Extracting and Managing Event Identity Information from Twitter}. $20^{th}$ International Conference on Information Quality, M.I.T, Boston (Notification Due: April 30, 2015)

\item \textbf{Debanjan Mahata}, John R. Talburt and Vivek Kumar Singh; \textit{From Chirps to Whistles : Discovering Event-specific Informative Content from Twitter}. Proceedings of the $7^{th}$ Annual ACM Web Science Conference. ACM, 2015, Oxford, England (Notification Due: April 30, 2015)

\end{itemize}
% Chapter 1

\chapter{Dissertation Overview} % Main chapter title

\label{overview} % For referencing the chapter elsewhere, use \ref{Chapter1} 

\lhead{Chapter 1. \emph{Dissertation Overview}} % This is for the header on each page - perhaps a shortened title

%\section{Social Media and Real-life Events}
% It has provided a communication platform to the masses enabling them to post short real-time messages in the form of micro-blogs, status updates, photographs and videos, to write full length articles expressing their views in blogs. This has turned the information consumers to original information producers and curators. According to a recent survey reported by Pew Research about 46\% of adult Internet users post original photos or videos online that they themselves have created \cite{pewresearch}. The humungous volumes of dynamic user-generated real-time data from social media provide great opportunities to businesses, governments, and researchers to tap valuable meaningful information for further analysis.

Social media has brought a paradigm shift in the way people communicate with each other. It has gone from being just a medium to a global medium of communication between people. Different types of social media platforms provide multiple venues to people for sharing first-hand experiences and exchange information about real-life events. It has become an indispensable source for disseminating news and real-time information about current events, using websites like Twitter, Facebook, Instagram, Flickr, Youtube, Vine, etc, that allow users to post short textual messages accompanied with images and videos. At the same time users also share their detailed citizen journalistic experiences in the form of diaries through different blogging platforms like Blogger, Wordpress, Medium, etc. Studies have shown the importance of different social media platforms as a news circulation service \cite{phelan2009using}, and a source for gauging public interest and opinions \cite{o2010tweets,singh2010clustering,singh2010mining,agarwal2012online}. It's efficacy as a real-time citizen-journalistic source of information has been recently harnessed in detection, extraction and analysis of real-life events \cite{sakaki2013tweet,popescu2011extracting,purohit2013twitris}. The activities of users producing content in social media has also been studied for gaining deep insights about how they group together to form communities around topics related to real-life events \cite{agarwal2013grouping,agarwal2014time,sen2012identifying}, and lead to collective action \cite{agarwal2014online,agarwal2012raising}.

With the popularity of social media there has been proliferation of unstructured textual content about different real-life events, in the Internet.
The information gained by identifying and tracking social media content expressing live reporting of an event, recent updates related to the event, insightful opinion about the different named entities (people, place, organization, etc) directly or indirectly involved with the event, summarization of content, among others, could prove to be extremely valuable for monitoring and gaining deeper actionable insights. There are tremendous applications in the areas of real-life event analysis, event management, opinion mining, reference tracking, online targeted marketing, recommendation engines, cyber security, enterprise data integration, among others. Thus, there is a need of a generic framework that has the following capabilities:
\begin{itemize}
\item can collect different types of textual content produced in social media related to an event
\item extract information that acts as an identity of the event used for characterizing it
\item maintain the extracted event identity information persistently for resolving constantly produced new content and discovering important event-specific information. 
\end{itemize}


The problem of collecting and extracting event identity information from social media is very similar to the task of event detection and tracking from newswires \cite{allan1998line,kumaran2004text}. However, in this thesis, we add new components of creating identity structures of an event and managing the tracked information persistently over time. In order to make our task well defined we avoid the task of detecting unidentified events, and instead track a pre-specified set of events. Also, the domain of social media poses additional challenges. News articles most often adhere to grammatical, syntactical and formal structures of writing, that are not common in the realm of social media. The user generated content in social media is most often colloquial, short, noisy and lack proper grammatical structures. This makes it a challenging task for the state-of-the-art natural language processing techniques to extract useful information and perform tasks like entity extraction and parts-of-speech tagging that lies at the core of the previous research on event detection and tracking.

The work presented in this thesis establishes the conceptual design and implementation of a framework capable of collecting, extracting and persistently managing event identity information from user generated textual content shared in social media (shown in Figure 1.1). The approach of the presented work is from the perspective of Entity Identity Information Management (EIIM) \cite{zhou2011entity}, with basic tenets of information quality at its core. Towards this objective, different challenges of mining high quality information from social media text is discussed and a patent pending novel approach to tackle the challenges for identifying event-specific informative content is explained, which lies at the heart of the framework. It further explores the applications of the research and concludes by pointing to different future directions of the work. 


%The thesis introduces the problem of Event Identity Information Management in social media, discusses the prevalent challenges and presents the implementation design of a framework capable of managing persistent identity information of pre-specified set of real-life events. It further explores the applications of the research and concludes by pointing to different future directions of the work. 

\begin{figure}[htbp]
\label{eiim}
  \caption{Event Identity Information Management (EIIM) Life Cycle for user generated textual content in social media}
  \centering
    \includegraphics[width=15.5cm,height=7cm]{Figures/EIIM.jpg}
\end{figure}

Some of the main contributions of the work are:

\begin{itemize}
\item Extending the Entity Identity Information Management model  \cite{zhou2011entity} from the closed world domain of Master Data Management (MDM) to the open and unstructured domain of social media.

\item Design and implementation of an \textit{Event Identity Information Management} framework that is capable of tracking and identifying event-specific information from long as well as short user generated textual content in social media. Towards this objective a data processing pipeline named \textit{Event Identity Information Management Life Cycle} is developed (Figure 1.1), which is capable of :
\begin{itemize}
\item collecting event related real-time content generated in social media
\item pre-processing them using natural language processing techniques
\item identifying high quality informative sources of information
\item extracting event-specific information in order to create \textit{Event Identity Information Structures} (EIIS) for persistently storing and characterizing the salient and high quality event related information 
\item identifying event-specific informative content produced in social media
\end{itemize}


\item Implementation of a supervised classifier in the domain of short and informal social media textual content, for segregating high quality informative messages having higher chances of containing event related information from the low quality non-informative ones. 

\item Analysis of informative and non-informative event related content from 3.8 million short textual social media messages.

\item A novel model that leverages mutually reinforcing relationships between blog posts and named entities mentioned in them, and simultaneously ranks blogs as well as the named entities, allowing identification of event-specific content and further analysis of event-specific information.

\item A novel model based on principle of mutual reinforcement that takes into account the semantics of relationships between short textual \textit{social media messages}, \textit{hashtags}, \textit{text units}, \textit{URLs} and \textit{users}, and represent them in a graph structure - \textit{EvenIdentitytInfoGraph}. A scalable graph processing iterative algorithm -\textit{EventIdentityInfoRank}, is implemented for ranking the nodes of the \textit{EventIdentityInfoGraph}. The algorithm is capable of simultaneously ranking \textit{social media messages}, \textit{hashtags}, \textit{text units}, \textit{URLs} and \textit{users} in terms of event-specific informativeness providing deeper insights into the identity of an event.

\item Evaluation of the proposed techniques against popularly used baseline techniques using large scale datasets.

\end{itemize}

Already published work as well as upcoming publications that represents our contributions related to specific topics covered by the broad area of research as presented in this thesis are given below. \\ 

\textbf{\LARGE Related Filed Patent}
\begin{itemize}
\item A System for Collecting, Ranking and Managing Entity Identity Information from Social Media (US 62135258). Inventors: \textbf{Debanjan Mahata} and John R. Talburt, Assignee: The Board Of Trustees Of The University Of Arkansas.
\end{itemize}

\textbf{\LARGE Related Award}
\begin{itemize}
\item \textbf{Debanjan Mahata} and John R. Talburt. \textit{Chatter that Matter : A Framework for Collecting, Extracting, and Managing Event Identity Information from Short Social Media Text}. Student Research and Creative Works Expo, Graduate Competition, University of Arkansas at Little Rock, April, 2015. (Awarded First Place in Engineering and Information Technology).  
\end{itemize}

\textbf{\LARGE Related Publications}
\begin{itemize}
\item \textbf{Debanjan Mahata}, John R. Talburt and Vivek Kumar Singh; \textit{Identifying and Ranking of Event-specific Entity-centric Informative Content from Twitter}. $20^{th}$ International Conference On Applications Of Natural Language To Information Systems (NLDB 2015), Passau, Germany. $17^{th}-19^{th}$ June, 2015.

\item \textbf{Debanjan Mahata} and John R. Talburt; \textit{A Framework for Collecting and Managing Entity Identity Information from Social Media}. $19^{th}$ International Conference on Information Quality, Xi'An, China.

\item \textbf{Debanjan Mahata} and Nitin Agarwal; \textit{Identifying Event-specific Sources from Social Media}. Online Social Media Analysis and Visualization. Lecture Notes in Social Networks, Springer, Kawash, Jalal (Ed). January, 2015.

\item Nitin Agarwal, \textbf{Debanjan Mahata}, and Huan Liu. \textit{Time-and Event-Driven Modeling of Blogger Influence}. Encyclopedia of Social Network Analysis and Mining. Springer New York, 2014. 2154-2165.


\item \textbf{Debanjan Mahata} and Nitin Agarwal. \textit{Learning from the crowd: An Evolutionary Mutual Reinforcement Model for Analyzing Events}. Advances in Social Networks Analysis and Mining (ASONAM), 2013 IEEE/ACM International Conference on. IEEE, 2013.

\item Nitin Agarwal, and \textbf{Debanjan Mahata}. \textit{Grouping the Similar among the Disconnected Bloggers}. Social Media Mining and Social Network Analysis: Emerging Research (2013), 54.

\item \textbf{Debanjan Mahata}, and Nitin Agarwal. \textit{What does everybody know? identifying event-specific sources from social media}. IEEE Fourth International Conference on Computational Aspects of Social Networks (CASoN), 2012.

\item \textbf{Debanjan Mahata} and Nitin Agarwal. \textit{Analyzing Event-specific Socio-Technical Behaviors Through the Lens of Social Media}. The International Sunbelt Social Network Conference (Sunbelt XXXII) organized by the International Network for Social Network Analysis (INSNA), March 12-18, 2012, Redondo Beach, California.

\item Vivek Kumar Singh, \textbf{Debanjan Mahata}, and Rakesh Adhikari. \textit{Mining the blogosphere from a socio-political perspective}. IEEE International Conference on Computer Information Systems and Industrial Management Applications (CISIM), 2010.

\item Vivek Kumar Singh, Rakesh Adhikari, and \textbf{Debanjan Mahata}. \textit{A clustering and opinion mining approach to socio-political analysis of the blogosphere}. IEEE International Conference on Computational Intelligence and Computing Research (ICCIC), 2010.

\end{itemize}

\textbf{\LARGE Related Submitted Publications}

\begin{itemize}
\item \textbf{Debanjan Mahata}, John R. Talburt, Vivek Kumar Singh and Rajesh Piryani; \textit{Chatter that Matter: A Framework for Identifying and Ranking Event-specific Informative Tweets}. $18^{th}$ International Conference on Text, Speech and Dialogue, Plzen, Czech Republic (Notification Due: May 10, 2015)

\item \textbf{Debanjan Mahata}, John R. Talburt and Vivek Kumar Singh; \textit{A Framework for Collecting, Extracting and Managing Event Identity Information from Twitter}. $20^{th}$ International Conference on Information Quality, M.I.T, Boston (Notification Due: April 30, 2015)

\item \textbf{Debanjan Mahata}, John R. Talburt and Vivek Kumar Singh; \textit{From Chirps to Whistles : Discovering Event-specific Informative Content from Twitter}. Proceedings of the $7^{th}$ Annual ACM Web Science Conference. ACM, 2015, Oxford, England (Notification Due: April 30, 2015)

\end{itemize}
  





 

%Twitter alone has 284 million monthly users,  posting 500 million tweets per day produces a variety of content\footnote{\tiny http://about.twitter.com/company}. A significant proportion of it are related to different real-life events (e.g, football matches, conferences, music shows, etc). Majority of this content are personal updates (e.g.  \textit{Thanks for the memories Sochi! I've had the time of my life \#Sochi2014 \#sochiselfie http://t.co/DqkLEaAMpo}), pointless babbles (e.g. \textit{Ted Cruz is a dangerous man. Crazy and gaining support. Megalomaniac leaders are bad, mkay. \#CPAC \#politics \#joke}) and spams (e.g \textit{New post: Sochi Was For Suckers - Laugh Studios/ http://t.co/cWQJCBp3Ow \#lol \#funny \#rofl \#funnypic \#wtf.}). Personal views and conversations might be of interest to a specific group of people. However, they are meaningless and provides no information to the general audience. On the other hand there are tweets that presents newsworthy content, recent updates and real-time coverage of on-going events (e.g. \textit{In \#Sochi, the Dutch are dominating the overall Olympic medal count http://t.co/jMR1WUqEK4 (Reuters) http://t.co/dAfDhEgTGA}). These tweets provide event-specific informative content and are more useful for general audience interested to know about the event. We call them as event-specific informative references. Table \ref{tweetsample} presents some examples of different types of tweets shared during real-life events.
%
%\begin{table}[htbp]
%\centering
%\caption{Examples of different event related tweets.}
%\label{tweetsample}
%     \begin{tabular}{|p{14cm}|} \hline
%     Ted Cruz is a dangerous man. Crazy and gaining support. Megalomaniac leaders are bad, mkay. \#CPAC \#politics \#joke [\textit{\textbf{personal/uninformative}}] \small \textit{\textbf{Event: `CPAC 2014'}}\\ \hline
%     Thanks for the memories Sochi! I've had the time of my life \#Sochi2014 \#sochiselfie http://t.co/DqkLEaAMpo. [\textit{\textbf{personal/uninformative}}] \small \textit{\textbf{Event: `Sochi Games'}} \\ \hline
%     \#SXSW14 \#SXSW \#sxswinteractive \#CPAC2014 \#CPAC \#CPACPickupLines \#CPACPanels Be squared away \@ perky TOP TWEETED of http://t.co/h0igdOVNW0. [\textit{\textbf{spam/uninformative}}] \small \textit{\textbf{Event: `CPAC 2014'}}\\ \hline
%In \#Sochi, the Dutch are dominating the overall Olympic medal count http://t.co/jMR1WUqEK4 (Reuters) http://t.co/dAfDhEgTGA. [\textit{\textbf{event-specific informative}}] \small \textit{\textbf{Event: `Sochi Games'}}\\ \hline
%New post: Sochi Was For Suckers - Laugh Studios/ http://t.co/cWQJCBp3Ow \#lol \#funny \#rofl \#funnypic \#fail \#wtf. [\textit{\textbf{spam/uninformative}}] \small \textit{\textbf{Event: `Sochi Games'}}\\ \hline
%It's \@tedcruz vs. \@SenJohnMcCain in a \#CPAC spat. What did they say? Find out on \#AC360 8p on \@CNN. [\textit{\textbf{event-specific informative}}] \small \textit{\textbf{Event: `CPAC 2014'}} \\ \hline
%     \end{tabular}
%\end{table}


%\section{Background : Entity Identity Information Management in Master Data Management}
%
%
%
%\section{Problem Definition and Research Questions}
%
%\section{General Challenges in Mining Social Media Text}
%
%\subsection{Information Overload}
%A daily average of 58 million tweets is posted in Twitter\footnote{http://www.statisticbrain.com/twitter-statistics/}.On an average 60 million  photos are shared in Instagram daily\footnote{http://instagram.com/press/}. Facebook stores 300 petabytes  of data related to its users from all over the world\footnote{http://expandedramblings.com/index.php/by-the-numbers-17-amazing-facebook-stats/}. These are some compelling statistics that makes social media not only rich in volume of data, but also variety, and the velocity at which data is being generated. Due to the great pace at which data is produced in social media, the search engines and content filtering algorithms often face the problem of information overload \cite{hemp2009death}. They suffer from the dilemma of assessing the accuracy and quality of information content in the sources being produced over their freshness. Thus, collecting different types of references of entities from various social media platforms, assessing their quality, resolving and extracting identity information of the entities poses great challenges in such a situation.
%
%\subsection{Veracity of Sources}
%Judging the accuracy of the information and deciding relevant information content in social media references for the purpose of extracting entity identity attributes constitutes another challenging situation. For trending topics the search engines have started showing real-time feeds from social media websites in their search results. This has attracted spammers who post trending hash-tags or keywords along with their spam content in order to attract people to their websites offering products or services \cite{benevenuto2010detecting}. An alarming 355\% growth of social spam has been reported in 2013\footnote{http://www.likeable.com/blog/2013/11/10-surprising-social-media-statistics/}. Social media has also been instrumental in spreading misinformation and rumors. Spread of misinformation not only results in pandemonium among the users\footnote{http://www.theguardian.com/uk/interactive/2011/dec/07/london-riots-twitter}  but also result in extraction of completely wrong information about entities.
%
%\subsection{Informal Text}
%Unlike sources of news media and edited documents on the web, the textual content of the social media sources are highly colloquial and pose great difficulties in extracting information. One of the most important sources of information about events, prevalent in the domain of social media are the micro-blogging platforms. Micro blogs pose additional challenges due to their brevity, noisiness, idiosyncratic language, unusual structure and ambiguous representation of discourse \cite{bontcheva2013twitie}. Variation in language, less grammatical structure of sentences, unconventional uses of capitalization, frequent use of emoticons, and abbreviations have to be dealt by any system processing social media content. Moreover, various signals of communications embedded in the text in the form of hash-tags (eg.\#sochi), retweets (RT) and user mentions (@) should be understood by the system in order to extract the contextual information hidden in the text. Intentional misspellings sometimes demonstrate examples of intonation in written text \cite{prevost1996information}. For instance, expressions like, `this is so cooool', emphasizes stress on the emotions and conveys more information that should be captured. It has been shown that it is extremely challenging for the state-of-the art information extraction algorithms to perform efficiently and give accurate results for micro-blogs \cite{derczynski2013microblog}. For example, named entity recognition methods typically show 85-90\% accuracy on longer texts, but 30-50\% on tweets \cite{ritter2011named}. Status messages in social networking websites, content in question answering websites, reviews, and discussions in blogs, and forums exhibit similar nature and present similar challenges to information extraction and text mining procedures.
%
%
%
%\subsection{Sampling Bias}
%Most commonly used method for obtaining data samples from social media websites is by using their application programming interfaces (APIs). Given the humungous amounts of data produced in real-time, the APIs cannot provide all the data to every single API requests. The requests are often made through a query interface by passing certain query parameters to the APIs. The amount of data returned against the queries may vary. This depends upon the popularity of the content related to the query. For example, in Twitter studies have estimated that by using Twitter's Streaming API users can expect to receive anywhere from 1\% of the tweets to over 40\% of tweets in near real-time\footnote{https://www.brightplanet.com/2013/06/twitter-firehose-vs-twitter-api-whats-the-difference-and-why-should-you-care/}. The only way to get access to all the tweets is to buy the firehose service, which is seldom done for academic purposes. Other real-time social media publishing services mostly follow the same model. Therefore, this might lead to biasness in the samples collected for studying event related phenomenon and for tracking all the important event related information being produced in real-time.
%
%\subsection{Multiple Data Sources}
%The APIs (Application Programming Interfaces) of the different social media websites returns data in different formats (JSON, XML) using different web standards (REST, HTTPS). Moreover, the information obtained from a social media website is dependent upon the type of content it produces. A video sharing website might return an entirely different set of information from a blogging website. Thus, integrating the data obtained from the various social media platforms for the purpose of extraction and tracking of event related information is also one of the challenges.
%
%\subsection{Lack of Evaluation Datasets}
%There is a lack of ground truth evaluation data for most of the social media text mining tasks. In traditional data mining research, there is often two types of datasets. One of them is known as training dataset and the other is known as test dataset. The models are trained or developed using the training datasets and are evaluated on test datasets. Thus, the test datasets act as the ground truth. The test dataset for various text mining tasks is mostly not available for social media data. It is often the duty of the researchers to create new test datasets in order to solve a specific task in social media. Sometimes this data might not be a benchmark dataset due to various unwanted noise and human error or perception in annotating the data. This might lead to wrong assumptions and false results.
%
%
%\section{Research Methodology}
%
%\section{Research Contributions}
%

The rest of the thesis is organized as follows:

Chapter \ref{events} gives an overview of the different social media websites and challenges in mining information from them. It also looks at the different perspectives of defining an event and gives the definition of events in social media as accepted by the presented work. Finally, it defines the problem of Event Identity Information Management from Social Media whose solution and application is extensively discussed throughout the rest of the thesis.

Chapter \ref{review} reviews the existing literature related to the topic of the thesis and highlights the challenges in applying previously available techniques to the domain of social media. It also discusses the similarities and dissimilarities of our work with the previous ones, and identifies the areas of our novel contributions that makes it different from the available techniques.

Chapter \ref{eiim} presents a detailed discussion of the \textit{Event Identity Information Management Life Cycle}, that is proposed as a solution to the problem that is solved in this thesis. It goes through all the components of the life cycle and gives a detailed explanation of the design choices, implementation and their working.

Chapter \ref{applications} highlights the potential real-life application of the \textit{Event Identity Information Management} framework implemented in this thesis. 

Chapter \ref{Conclusion} draws conclusions of the work presented in this thesis and points to future directions of the work.

% Chapter 3

\chapter{Social Media and Real-life Events} % Main chapter title

\label{events} % For referencing the chapter elsewhere, use \ref{Chapter1} 

\lhead{Chapter 2. \emph{Social Media and Real-life Events}} % This is for the header on each page - perhaps a shortened title

\section{Social Media}

\section{General Challenges in Social Media Mining}

\section{Events from Different Perspectives}

\subsection{Topic Detection and Tracking}

\subsection{Automatic Content Extraction}

\subsection{Multimedia Event Detection}

\section{Events in Social Media}

\section{Problem of EIIM in Social Media} 
% Chapter 3

\chapter{Literature Review} % Main chapter title

\label{review} % For referencing the chapter elsewhere, use \ref{Chapter1} 

\lhead{Chapter 3. \emph{Literature Review}} % This is for the header on each page - perhaps a shortened title

\section{Event Identification in News Text}
The event detection task \cite{allan2002topic} in the TDT program (Topic Detection and Tracking), led to significant advancements in the field of event-based organization of broadcast news. Some of the efforts in the TDT program focused on online event detection from continuous and real-time streams of textual news documents in newswires \cite{allan1998line,kumaran2004text}. While others explored the detection of past events from archived news documents \cite{yang1998study}. 

The textual content in news documents are different from the short informal text common in the realm of social media.  Most of these documents contain formal text with well-formed grammatical structures, enabling the researchers to rely on the state-of-the-art natural language processing techniques. Named entity extraction and Parts-of-Speech (POS) tagging are among the widely used techniques. Zhang et al. \cite{zhang2007new} extracted named entities and POS tags from textual news documents, and used them to reweigh tf-idf representations of these documents for the new event detection task. Filatova and Hatzivassiloglou \cite{hatzivassiloglou2003domain} identified named entities corresponding to participants, locations, and times in text documents, and then used the relationships between certain types of entity pairs to detect event content. Hatzivassiloglou et al. \cite{hatzivassiloglou2000investigation} used linguistic features (e.g., noun phrase heads, proper names) and learned a logistic regression model for combining these features into a single similarity value. Makkonen et al. \cite{makkonen2004simple} extracted meaningful semantic features such as names, time references, and locations, and learned a similarity function that combines these metrics into a single clustering solution. They concluded that augmenting documents with semantic terms did not improve performance, and reasoned that inadequate similarity functions were partially to blame. 

Extracting events from text has been the focus of numerous studies as part of the NIST initiative for Automatic Content Extraction (ACE) \cite{ahn2006stages,ji2008refining}. The ACE program defines event extraction as a supervised task, given a small set of predefined event categories and entities, with the goal of extracting a unified representation of the event from text
via attributes (e.g., type, subtype, modality, polarity) and event roles (e.g., person, place, buyer, seller). Ahn \cite{ahn2006stages} divided the event extraction task into different subtasks, including identification of event keyword triggers (see Chapter 2), and determination of event
coreference, and then used machine learning methods to optimize and evaluate the results of each subtask. Ji and Grishman \cite{ji2008refining} proposed techniques for extracting event content from multiple topically similar documents, instead of the traditional approach of extracting events from individual documents in isolation. In contrast with the predefined templates outlined by ACE, Filatova et al. \cite{filatova2006automatic} presented techniques to automatically create templates for event types, referred to as domains, given a set of domain instances (i.e., documents containing information related to events that belong to the domain). 

As already discussed, social media documents are extremely concise, noisy and lacks well-established grammatical structures. Therefore, the techniques used in these works are not suitable for identification of events from social media.  It has been shown that it is extremely challenging for the state-of-the art information extraction algorithms to perform efficiently and give accurate results for micro-blogs \cite{derczynski2013microblog}. For example, named entity recognition methods typically show 85-90\% accuracy on longer texts, but 30-50\% on tweets \cite{ritter2011named}. Therefore, new approaches had to be taken, leading to new techniques for detecting events in social media, which we discuss next.

%One of the goals of the EIIM framework presented in this thesis, is to identify and track event related content being generated in social media. However, the framework does not require detection of unknown events from real-time streams of social media messages. Instead it is provided with a predefined set of events along with predefined hashtags for the respective events, which are used for relevant data collection.

\section{Event Identification in Social Media}
While event detection in textual news documents has been studied in depth, the identification
of events in social media sites is still in its infancy. Several related papers explored the unknown event identification scenario in social media.
Weng and Lee \cite{weng2011event} proposed wavelet-based signal detection techniques for identifying
real-life events from Twitter. These techniques can detect significant bursts or trends
in a Twitter data stream. Sankaranarayanan et al. \cite{sankaranarayanan2009twitterstand}
identified late breaking news events on Twitter using clustering, along with a text-based
classifier and a set of handpicked news seeders. But they do not take into account the filtering of non-event content, which results in poor performance. Segregating the messages that have high likelihood of containing informative content from the ones with chances of having non-informative content are at the core of our work.  

% but, unlike our work in Chapter 4, they do not filter the vast
%amount of non-event content that exists on Twitter. This, unfortunately, results in poor
%performance, with very low precision scores compared with the precision achieved by our
%methods. Related to our work in Chapters 4 and 5, 



%As we discussed, such text-based and seeder-driven filtering of
%non-event data can be used to generate the event document stream we use in Chapter 5.

Petrovic et al. \cite{petrovic2010streaming} used locality-sensitive hashing to detect the first tweet associated with an event in a stream of Twitter messages. Rattenbury et al. \cite{rattenbury2007towards} analyzed the temporal usage distribution of tags to identify tags that correspond
to events. Chen and Roy \cite{chen2009event} used the time and location associated with Flickr image
tags to discover event-related tags with significant distribution patterns (e.g.bursts) in
both of these dimensions.
 
%
%%We use the general text-based classifier suggested in \cite{sankaranarayanan2009twitterstand} and a method for identifying top events suggested by Petrovic et
%%al. \cite{petrovic2010streaming} as baseline approaches in our evaluation of the unknown identification methods
%%of Chapter 4. While our work in the unknown event identification scenario focuses on timely, online, analysis, several efforts tried to address this task using retrospective analysis. 


Recent efforts proposed techniques for known identification of events in social media.
Many of these techniques rely on a set of manually selected terms to retrieve event-related
documents from a single social media site \cite{sakaki2010earthquake,yardi2010tweeting}. Our method of tracking an event is similar to this. We also use predefined hashtags to bootstrap the process of collecting data related to a known set of events. Sakaki et al. \cite{sakaki2010earthquake} developed
techniques for identifying earthquake events on Twitter by monitoring keyword triggers
(e.g., earthquake or shaking). In their setting, the type of event must be known a
priori, and should be easily represented using simple keyword queries. Benson et al. \cite{benson2011event} identified Twitter messages for concert events using statistical models to automatically tag artist and venue terms in Twitter messages.
Their approach is novel and fully automatic, but it limits the set of identified messages for
concert events to those with explicit artist and venue mentions. Importantly, both of these
approaches are tailored to one specific social media site. In contrast, we propose methods
for identifying social media documents across multiple sites with varying types of documents
(e.g., photos, videos, textual messages). Our goal is to automatically retrieve social media
documents for any planned event, without any assumption about the textual content of
the event or its associated documents. While not exclusively in the social media domain,
Tsagkias et al. \cite{tsagkias2011linking} extracted named entities and quotations from news articles, as
well as explicit links between news and social media documents, to identify social media
utterances related to individual news stories. In contrast with their well formed, lengthy
textual documents and explicitly linked content, content in our known event identification
setting (Chapter 6) is brief and often noisy, and generally does not contain explicit links to
social media documents.

\section{Information Quality in Social Media}

\section{Ranking and Summarization of Short Textual Social Media Posts}
There are many web hosted applications that supplements the default search provided by Twitter in order to effectively retrieve relevant and high quality tweets from different perspectives\footnote{\tiny http://mashable.com/2009/04/22/twitter-search-services}. On going through these services we found that the most commonly used criteria for ranking tweets are recency, popularity based on retweets and favorite counts, authority of the users posting the tweets and content relevance. Twitter itself uses the popularity of the tweets and features mined from the profile of the users in order to provide personalized search results ordered by recency\footnote{\tiny https://blog.twitter.com/2011/engineering-behind-twitter\%E2\%80\%99s-new-search-experience}. A study of different state-of-the-art features and approaches commonly used for ranking tweets has been documented by \cite{damak2013effectiveness, nagmoti2010ranking}. Seen\footnote{\tiny http://seen.co} is a new state-of-the-art platform that uses a proprietary algorithm named \textit{SeenRank} for ranking event related tweet content for presenting event highlights and summaries. In this work, we consider \textit{SeenRank} as one of our baselines. As the number of retweets of a tweet is widely used for ranking, we also use it as one of our baselines. In the context of our work we name the ranking scheme as \textit{RTRank}

Apart from the existing real-world search applications, several adaptations of \textit{PageRank} \cite{page1999pagerank} has been proposed by the scientific community for ranking tweets and users in Twitter \cite{weng2010twitterrank,tunkelang2009twitter, hallberg2012adaptation}. 
%TweetRank \cite{hallberg2012} is one such adaptation that ranks tweets by taking into account the direct relationships between tweets in the form of retweets and replies, as well as indirect follower-friend relationships, and usage of similar hashtags. 
Various learning to rank approaches have been used for ordering tweets retrieved for a given query in terms of their relevance and quality \cite{duan2010empirical,mccreadie2013relevance,vosecky2012searching}. None of these ranking techniques have been devised for event-specific content. An attempt to solve a similar problem presented in this paper was made by \cite{becker2011selecting}. They represented tweets of an event in a cluster and calculated the similarity of individual tweets with the centroid of the cluster. Then they ranked the tweets based on the decreasing value of their similarity. We use this approach as one of our baselines.

%Research on summarizing, discovering, or otherwise presenting social media content has
%gathered recent attention. The task of social media summarization is related to our content
%, but instead of selecting a set of potentially disconnected
%messages, it aims to construct a coherent summary representation. Several eorts [CP11;
%SHK10] considered ways to summarize a set of social media documents related to a specific
%topic. Sharifi et al. [SHK10] proposed approaches for summarizing a set of Twitter messages
%that were retrieved in response to a keyword query. They used graph-based phrase
%reinforcement and tf-idf techniques to produce very short summaries, which often consist
%of fewer than 10 words. Chakrabarti and Punera [CP11] proposed techniques for summarizing Twitter messages for events. They used Hidden Markov Models to segment the set
%of messages into sub-events, and then selected key messages from each interesting subevent,
%to include in the overall summary. This approach, as the authors note, is geared
%towards structured, long-running events and its effectiveness has not been determined for
%short events such as concerts or festivals.

\section{Reference Tracking and Entity Resolution}

An entity in general may be defined as an object that has a distinct, independent and self-contained existence, whether hypothetical or real. Thus, an entity could be a person (e.g `Barack Obama'), a place (e.g `Little Rock'), a product (e.g `Iphone6'), an event (eg `Egyptian Revolution') or anything from real-life that has an individual identity. The identity of an entity is a set of attribute values for that entity along with a set of distinct rules that allow that entity to be distinguished from all other entities of the same class in a given context [3].


\begin{figure}[htbp]
  \caption{Identity Integrity component of the EIIM life cycle.}
  \centering
    \includegraphics[width=14cm,height=7cm]{Figures/OriginalEIIM.jpg}
\end{figure} 

In this section we explain the current EIIM process that lays the foundation and acts as a background of the presented research.
The idea of Entity Identity Information Management (EIIM) as defined by [15] is the collection and management of identity information of real-world entities with the goal of sustaining entity identity integrity. Their model of EIIM was motivated by the problem of entity resolution in information systems, particularly in the domain of MDM (Master Data Management). They define entity resolution as the process of determining whether two references to real-world objects in an information system are referring to the same object, or to different object [16]. The EIIM life cycle as proposed by them is an iterative process that combines entity resolution and data structures representing entity identity into specific operational configurations (EIIM configurations, as shown in Figure 3), that when executed in concert, work to maintain the entity identity integrity of master data over time. The EIIM framework is implemented by developing open source software known as OYSTER .



% Chapter 4

\chapter{Event Identity Information Management Life Cycle} % Main chapter title

\label{eiim} % For referencing the chapter elsewhere, use \ref{Chapter1} 

\lhead{Chapter 4. \emph{Event Identity Information Management Life Cycle}} % This is for the header on each page - perhaps a shortened title

\section{Identity Integrity}

\begin{figure}[htbp]
  \caption{Identity Integrity component of the EIIM life cycle.}
  \centering
    \includegraphics[width=10cm,height=9cm]{Figures/EIIMComponents/IdentityIntegrity.jpg}
\end{figure}

\section{Event Reference Collection}

\begin{figure}[htbp]
  \caption{Event Reference Collection component of the EIIM life cycle.}
  \centering
    \includegraphics[width=10cm,height=9cm]{Figures/EIIMComponents/EventReferenceCollection.jpg}
\end{figure}

\section{Event Reference Preparation}

\begin{figure}[htbp]
  \caption{Event Reference Preparation component of the EIIM life cycle.}
  \centering
    \includegraphics[width=10cm,height=9cm]{Figures/EIIMComponents/EventReferencePreparation.jpg}
\end{figure}

\section{Event Information Quality}

\begin{figure}[htbp]
  \caption{Event Information Quality component of the EIIM life cycle.}
  \centering
    \includegraphics[width=10cm,height=9cm]{Figures/EIIMComponents/EventInformationQuality.jpg}
\end{figure}

\section{Event Identity Information Capture}

\begin{figure}[htbp]
  \caption{Event Identity Information Capture component of the EIIM life cycle.}
  \centering
    \includegraphics[width=10cm,height=9cm]{Figures/EIIMComponents/EventIdentityInformationCapture.jpg}
\end{figure}

\section{Event Identity Information Structure}

\begin{figure}[htbp]
  \caption{Event Identity Information Structure component of the EIIM life cycle.}
  \centering
    \includegraphics[width=10cm,height=9cm]{Figures/EIIMComponents/EventIdentityInformationStructure.jpg}
\end{figure}

\section{Event Identity Information Processing\label{EventIdentityInformationProcessing}}

\begin{figure}[htbp]
  \caption{Event Identity Information Processing component of the EIIM life cycle.}
  \centering
    \includegraphics[width=10cm,height=9cm]{Figures/EIIMComponents/EventIdentityInformationProcessing.jpg}
\end{figure}

\section{Event Reference Resolution}

\begin{figure}[htbp]
  \caption{Event Reference Resolution component of the EIIM life cycle.}
  \centering
    \includegraphics[width=10cm,height=9cm]{Figures/EIIMComponents/EventReferenceResolution.jpg}
\end{figure}

\section{Event Analytics}
\begin{figure}[htbp]
  \caption{Event Analytics component of the EIIM life cycle.}
  \centering
    \includegraphics[width=10cm,height=9cm]{Figures/EIIMComponents/EventAnalytics.jpg}
\end{figure} 
% Chapter 5

\chapter{Discovering Event-specific Informative Content from Twitter} % Main chapter title

\label{TwitterStudy} % For referencing the chapter elsewhere, use \ref{Chapter1} 

\lhead{Chapter 5. \emph{Discovering Event-specific Informative Content from Twitter}} % This is for the header on each page - perhaps a shortened title

\section{Twitter and Event Related Content}

\section{Analysis of Informative and Non-informative Content in Tweets}

\section{EventIdentityInfoGraph}

\section{EventIdentityInfoRank}

\section{Experiments}

\subsection{Data Collection}

\subsection{Data Preparation}

\subsection{Baselines}

\subsection{Evaluation}

\subsection{Sample Results} 
% Chapter 6

\chapter{Conclusion and Future Work} % Main chapter title

\label{Conclusion} % For referencing the chapter elsewhere, use \ref{Chapter1} 

\lhead{Chapter 6. \emph{Conclusion and Future Work}} % This is for the header on each page - perhaps a shortened title

\section{Conclusion}

\section{Future Work}

\subsection{Summarizing Event Related Content}

\subsection{Identifying Insightful Opinionated Content Related to Events}

\subsection{Event Topic Modeling}

\subsection{Event-specific Recommendations}

\subsection{Distributed Processing of EventIdentityInfoGraph}

\subsection{Event Ontology for Social Media} 
\input{Chapters/Chapter7} 

%----------------------------------------------------------------------------------------
%	THESIS CONTENT - APPENDICES
%----------------------------------------------------------------------------------------

\addtocontents{toc}{\vspace{2em}} % Add a gap in the Contents, for aesthetics

\appendix % Cue to tell LaTeX that the following 'chapters' are Appendices

% Include the appendices of the thesis as separate files from the Appendices folder
% Uncomment the lines as you write the Appendices

% Appendix A

\chapter{Appendix Title Here} % Main appendix title

\label{AppendixA} % For referencing this appendix elsewhere, use \ref{AppendixA}

\lhead{Appendix A. \emph{Twitter Data Collection}} % This is for the header on each page - perhaps a shortened title

Write your Appendix content here.
%\input{Appendices/AppendixB}
%\input{Appendices/AppendixC}

\addtocontents{toc}{\vspace{2em}} % Add a gap in the Contents, for aesthetics

\backmatter

%----------------------------------------------------------------------------------------
%	BIBLIOGRAPHY
%----------------------------------------------------------------------------------------

\label{Bibliography}

\lhead{\emph{Bibliography}} % Change the page header to say "Bibliography"

\bibliographystyle{unsrtnat} % Use the "unsrtnat" BibTeX style for formatting the Bibliography

\bibliography{Bibliography} % The references (bibliography) information are stored in the file named "Bibliography.bib"

\end{document}  